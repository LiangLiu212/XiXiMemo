\section{Parameters estimation}
\subsection{Maximum likelihood fit}

A maximum likelihood fit is used to extract the decay parameters, which are only 
related to the helicity angles. For each channel, the probability distribution 
function (PDF) with eight unknow parameters $\Omega$ can be defined in nine dimensions
helicity angles spanned by $\xi$:
\begin{equation}
	\mathcal{P}(\xi;\Omega) = \mathcal{W}(\xi;\Omega)\varepsilon(\xi)/\mathcal{N}(\Omega)
\end{equation}
The full likelihood function can be written as 
\begin{equation}
	\mathcal{S} = -\sum_{i = 1}^N\ln \mathcal{W}(\xi_i;\Omega) + 
	\sum_j \sum_i^{N_j^{\rm bkg}}\ln \mathcal{W}(\xi_i;\Omega)
	+ N^{\rm signal} \times \ln \mathcal{N}(\Omega) - \sum_i^N \ln \varepsilon(\xi_i)
\end{equation}

\begin{itemize}
	\item Helicity angles and parameters

		For neutron channel, the helicity angles $\xi = (\theta_{\Xi}, \theta_{\Lambda}, \phi_{\Lambda}, 
		\theta_{\bar{\Lambda}}, \phi_{\bar{\Lambda}}, \theta_{n}, \phi_{n}, 
		\theta_{\bar{p}}, \phi_{\bar{p}})$ 
		and the parameters $\Omega = (\alpha_{J/\psi}, \Delta \Phi, \alpha_{\Xi^-},
		\phi_{\Xi^-}, \alpha_{\bar{\Xi}^+}, \phi_{\bar{\Xi}^+}, \alpha_{\Lambda}^0,
		\alpha_{\bar{\Lambda}}^+)$;

		For anti-neutron channel, the helicity angles 
		$\xi = (\theta_{\Xi}, \theta_{\Lambda}, \phi_{\Lambda}, 
		\theta_{\bar{\Lambda}}, \phi_{\bar{\Lambda}}, \theta_{p}, \phi_{p}, 
		\theta_{\bar{n}}, \phi_{\bar{n}})$ 
		and the parameters $\Omega = (\alpha_{J/\psi}, \Delta \Phi, \alpha_{\Xi^-},
		\phi_{\Xi^-}, \alpha_{\bar{\Xi}^+}, \phi_{\bar{\Xi}^+}, \alpha_{\Lambda}^-,
		\alpha_{\bar{\Lambda}}^0)$.
	\item $-\sum_{i = 1}^N\ln \mathcal{W}(\xi_i;\Omega)$

		The first item is the sum of all events (total $N$ events) which survive the 
		final selection criteria in experiment data. $\mathcal{W}(\xi_i;\Omega)$ is 
		the amplitude value for $i$-th events. In practice, we will try to find 
		minimum value of a function. So, there is a negative sign in the PDF.
	\item $\sum_j \sum_i^{N_j^{\rm bkg}}\ln \mathcal{W}(\xi_i;\Omega)$

		The second item stands for the background subtracting. There are total
		$j = 4$ types of background: mis-combination background, charged decay
		channel, $J/\psi \to \gamma \eta_c$ and non-peaking background. $N_j^{\rm bkg}$
		is the estimated number of events for $j$-th type of background.

	\item $N^{\rm signal} \times \ln \mathcal{N}(\Omega)$

		$\mathcal{N}(\Omega)$ is the normalization factor given by:
		\begin{equation}
			\mathcal{N}(\Omega) = \int \mathcal{W}(\xi;\Omega)\varepsilon(\xi) {\rm d} \xi 
		\end{equation}
		In practice, there are two method to calculate the normalization factor,
		using PHSP MC or mDIY MC, propagated through the detector and reconstructed as
		data. The normalization factor calculated with PHSP MC is approximately given 
		as:
		\begin{equation}
			\mathcal{N}(\Omega) = \frac{1}{M} \sum_{j=1}^M \mathcal{W}(\xi_j;\Omega)	
		\end{equation}
		By using mDIY MC, the normalization factor is given as:
		\begin{equation}
			\mathcal{N}(\Omega) = \frac{1}{M} \sum_{j=1}^M \frac{\mathcal{W}(\xi_j;\Omega)	}{\mathcal{W}(\xi_j;\Omega_0)}
		\end{equation}
		where $M$ is the total number of events of the corresponding MC sample that  
		survived all the event selection criteria as data and with additional photon 
		matched angle requirement..

		{\bf Comments on normalization factor}

	\item $- \sum_i^N \ln \varepsilon(\xi_i)$

	 	The last item is the efficiencies which is not dependent on the parameters
		in $\Omega$ and will only affects the overall log-likelihood normalization.
		Therefore, this item is a constent and can be dropped.


\end{itemize}

\subsection{Input-Output check}

An input-output check is performed to guarantee the correctness of fitting procedure. 
30 sets of mDIY MC is used to do the input-output check and the number of events of
each set is equal to the signal yield from data. The backgrounds are also considered
by including the background events from the inclusive and exclusive MC and then using 
the same method as used in data to subtract the background. Table~\ref{} lists the
input-output value. The distributions of the fitting results and the pull distribution
are shown in Fig.~\ref{}. It is clear to see that the output values are consistent with
the input.

