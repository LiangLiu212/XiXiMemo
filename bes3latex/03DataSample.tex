\section{Data Sample}

\subsection{BESIII Detector}
The BESIII detector~\cite{Ablikim:2009aa} records symmetric $e^+e^-$ collisions
provided by the BEPCII storage ring~\cite{Yu:IPAC2016-TUYA01}, which operates with a peak luminosity of $1\times10^{33}$~cm$^{-2}$s$^{-1}$
in the center-of-mass energy range from 2.0 to  4.95~GeV.
BESIII has collected large data samples in this energy region~\cite{Ablikim:2019hff}. The cylindrical core of the BESIII detector covers 93\% of the full solid angle and consists of a helium-based
 multilayer drift chamber~(MDC), a plastic scintillator time-of-flight
system~(TOF), and a CsI(Tl) electromagnetic calorimeter~(EMC),
which are all enclosed in a superconducting solenoidal magnet
providing a 1.0~T  (0.9~T in
2012) magnetic field. The solenoid is supported by an
octagonal flux-return yoke with resistive plate counter muon
identification modules interleaved with steel.
%The acceptance of charged particles and photons is 93\% over $4\pi$ solid angle.
The charged-particle momentum resolution at $1~{\rm GeV}/c$ is
$0.5\%$, and the
${\rm d}E/{\rm d}x$
resolution is $6\%$ for electrons
from Bhabha scattering. The EMC measures photon energies with a
resolution of $2.5\%$ ($5\%$) at $1$~GeV in the barrel (end cap)
region. The time resolution in the TOF barrel region is 68~ps, while
that in the end cap region is 110~ps.  The end cap TOF
system was upgraded in 2015 using multigap resistive plate chamber
technology, providing a time resolution of
60~ps~\cite{etof}.


\subsection{Data sample}
From 2009 to 2019, over $10^{10}$ $J/\psi$ events were collected with BESIII detector. 
In 10 years, the data sets were taken at four separate time regions, donated 2009, 2012,
2018 and 2019.
The number of $J/\psi$ events is determined by using inclusive decay of the $J/\psi$.
Table~\ref{tab:jpsievent} shows the number of $J/\psi$ events in each data sets. Due to the variation of 
detector status and reconstruction efficiency, we will perform the analysis for each data 
sets separately.

\begin{table}[hbtp]
	\centering
	\normalsize
	\caption[]{{The number of events for $J/\psi$ data sets. }}
	\label{tab:jpsievent}
\begin{tabular}{lr}
\firsthline
Data sets    & Number of events \\
\hline
	2009 & $(224.0\pm1.3)\times10^6$ \\
	2012 & $(1088.5\pm4.4)\times10^6$ \\
	2018 & \multirow{2}{*}{$(8774.0\pm39.4)\times10^6$} \\
	2019 &  \\
	\hline
	Total & $(1088.5\pm4.4)\times10^6$ \\
\lasthline
\end{tabular}
\end{table}


\subsection{Monte Carlo Simulation}
\label{sec:mcsimulation}
\subsubsection{Inclusive Monte Carlo}

Simulated data samples produced with a {\sc
geant4}-based~\cite{geant4} Monte Carlo (MC) package, which
includes the geometric description of the BESIII detector and the
detector response, are used to determine detection efficiencies
and to estimate backgrounds. The simulation models the beam
energy spread and initial state radiation (ISR) in the $e^+e^-$
annihilations with the generator {\sc
kkmc}~\cite{ref:kkmc}. 

\begin{itemize}
\item {  the $J/\psi$ data set \\
The inclusive MC sample includes both the production of the $J/\psi$
resonance and the continuum processes incorporated in {\sc
		kkmc}~\cite{ref:kkmc}.}
\end{itemize}
All particle decays are modelled with {\sc
evtgen}~\cite{ref:evtgen} using branching fractions 
either taken from the
Particle Data Group~\cite{pdg}, when available,
or otherwise estimated with {\sc lundcharm}~\cite{ref:lundcharm}.
%{\it [ORIGINAL:
%The known decay modes are modelled with {\sc
%evtgen}~\cite{ref:evtgen} using branching fractions taken from the
%Particle Data Group~\cite{pdg}, and the remaining unknown charmonium decays
%are modelled with {\sc lundcharm}~\cite{ref:lundcharm}.] }
Final state radiation~(FSR)
from charged final state particles is incorporated using the {\sc
photos} package~\cite{photos}.

\subsubsection{Signal Monte Carlo}
The following Monte Carlo samples are alse been generated by ourselves.

Phase space (PHSP MC) for two decay channel were generated for calculating 
the normalization in the maximum log likelihood method.

Signal MC samples simulated using the parameters estimated from data (mDIY MC)
were generated as a control sample searching for inconsistencies between
data and MC and used for input/output check and selection criteria optimization.
These values  are within the fit uncertainties of the experimentally obtained 
values and CP-conservation is assumed. The true distributions of the momentum of 
final states and 16 production moments are ploted in Fig.~\ref{fig:tru16momxixipm},~\ref{fig:tru16momxixipp},~\ref{fig:trufmomxixipm},~\ref{fig:trufmomxixipp}.
(For the four sets of MC simulation, the distributions are very similar. We only use 2009 MC sets as 
an example to show the distributions. The distributions of others can be found in Appendix.~\ref{app:trueinfo}.)

\begin{figure*}[hp]
  \centering
  \mbox
  {
  \begin{overpic}[width=0.8\textwidth]{Figure/xixipmmom2009.eps}
  \end{overpic}
  }
	\caption{Using 2009 mDIY MC as example to show the true distributions of the 16 moments of neutron channel for $J/\psi\to \Xi^- \bar{\Xi}^+$.} 
 \label{fig:tru16momxixipm}
\end{figure*}



\begin{figure*}[hp]
  \centering
  \mbox
  {
  \begin{overpic}[width=0.8\textwidth]{Figure/xixippmom2009.eps}
  \end{overpic}
  }
 \caption{Using 2009 mDIY MC as example to show the true distributions of the 16 moments of anti-neutron channel for $J/\psi\to \Xi^- \bar{\Xi}^+$.}
 \label{fig:tru16momxixipp}
\end{figure*}


\begin{figure*}[hp]
  \centering
  \mbox
  {
  \begin{overpic}[width=0.8\textwidth]{Figure/xixipmfmom2009.eps}
  \end{overpic}
  }
 \caption{Using 2009 mDIY MC as example to show the true momentum of final states and $\Lambda$ resonances in neutron channel for $J/\psi\to \Xi^- \bar{\Xi}^+$ The order of final states in legend is 
	$\Lambda$, $\bar{\Lambda}$, $n$, $\bar{p}$, $\pi^+_{\Xi}$, $\pi^+_{\Lambda}$, $\pi^-_{\Xi}$, and $\pi^0$.}
 \label{fig:trufmomxixipm}
\end{figure*}

\begin{figure*}[hp]
  \centering
  \mbox
  {
  \begin{overpic}[width=0.8\textwidth]{Figure/xixippfmom2009.eps}
  \end{overpic}
  }
 \caption{Using 2009 mDIY MC as example to show the true momentum of final states and $\Lambda$ resonances in anti-neutron channel for $J/\psi\to \Xi^- \bar{\Xi}^+$ The order of final states in legend is 
	$\Lambda$, $\bar{\Lambda}$, $p$, $\bar{n}$, $\pi^-_{\Xi}$, $\pi^-_{\Lambda}$, $\pi^+_{\Xi}$, and $\pi^0$.}
 \label{fig:trufmomxixipp}
\end{figure*}




The number of events of PHSP MC and mDIY MC samples are decided according to
the number of $J/\psi$ events and the branching fraction 
$\mathcal{B}(J/\psi \to \Xi^-\bar{\Xi}^+) = (9.7\pm0.8)\times10^{-4}$, 
$\mathcal{B}(\Xi^- \to \Lambda \pi^-) = (99.887\pm0.035)\%$,
$\mathcal{B}(\Lambda \to p \pi^-) = (63.9\pm0.5)\%$, and
$\mathcal{B}(\Lambda \to n \pi^0) = (35.8\pm0.5)\%$.
There is no doubt that the more statistic the MC has, the better for analysis.
Taking CPU time into account, we decide to generate a PHSP MC sample and a mDIY 
MC sample with 30 times the corresponding experimental data statistic. The number of 
event of MC samples for differents years are listed in Table.~\ref{tab:mcevent}.
\begin{table}[hbtp]
	\centering
	\normalsize
	\caption[]{{The number of events for PHSP MC and mDIY MC samples. }}
	\label{tab:mcevent}
\begin{tabular}{lrr}
\firsthline
	Data sets    & PHSP MC (million) & mDIY MC (million)  \\
\hline
	2009 & 1.8 & 1.8 \\
	2012 & 9 & 9 \\
	2018 & 37.8 & 37.8 \\
	2019 & 37.8 & 37.8 \\
	\hline
	Total & 86.4 & 86.4 \\
\lasthline
\end{tabular}
\end{table}



