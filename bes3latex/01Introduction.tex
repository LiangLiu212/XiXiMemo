\section{Introduction}

From the fundamental point of view, all matter is built out of fermions (spin-1/2 particles) 
with quarks being the elementary constituents. The antimatter in the modern theory was first 
predicted by Paul Dirac and discovered by Carl D. Anderson. Cosmological observations tell us
that our universe contains more matter than antimatter. The asymmetry between matter and 
antimatter can be characterized in terms of the baryon-to-photon ratio
\begin{equation}
	\eta \equiv \frac{n_{B} - n_{\bar{B}}}{n_{\gamma}} \approx 6\times10^{-10}.
\end{equation}
The physical process responsible for the asymmetry is called baryogengesis. To discover
the mechanism behind baryogengesis is one of the most important unresolved problems in 
fundamental physics. For now, it is well know that there necessary conditions for 
baryogengesis which is called Sakharov's conditions.
\begin{enumerate}
	\item B violation
	\item Loss of thermal equilibrium
	\item C, CP violation
\end{enumerate}

In 1956, the idea of testing the violation of parity (P) symmetry was proposed by 
Tsung-Dao Lee and Chen-Ning Yang firstly. The product of two transformations
charge conjugation (C) and parity (P) is the true symmetry between matter and 
antimatter. The Kobayashi-Maskawa mechanisam is the only confirmed way of CP violation predicted
by the Standard Model. The measuremet of the meson decays show that the Kobayashi-Maskawa
is, very likely, the dominant source of CP violation in low-energy flavor-changing 
processes. However, it predicts present baryon number density that is many orders of
magnitude below the cosmological observations, which indicates that there must exist
sources of CP violation beyong the Kobayashi-Maskawa phase in our Universe.

The hadronic decay of hyperons proceeds into both parity-violating ($S$-wave) and 
parity-conserving ($P$-wave) final states with amplitudes $S$ and $P$. The amplitude can
be written as
\begin{equation}
	{\rm Amp}(B\to b\pi) = S + P {\bf \sigma} \cdot {\bf q}
\end{equation} 
The experimental observables are the total decay width $\Gamma$, and the normalized decay asymmetry 
parameters $\alpha$, $\beta$, and $\gamma$.
\begin{equation}
	\begin{aligned}
		&\alpha^2 + \beta^2 +\gamma^2 = 1, \\
		&\alpha = 2 {\rm Re} (S^*P) / (|S|^2 + |P|^2), \\
		&\beta = 2 {\rm Im} (S^*P) / (|S|^2 + |P|^2). \\
	\end{aligned}
\end{equation}
Only two of these three parameters are independent.
 So, the decay parameters are usually parametrized by using two 
 essentially independent parameters $\alpha$ and $\phi$
\begin{equation}
	\begin{aligned}
		\beta &= \sqrt{1-\alpha^2} \sin \phi,\\
		\gamma &= \sqrt{1-\alpha^2} \cos \phi,
	\end{aligned}
\end{equation}
which are more closely related to experimental measurement. Using $\alpha$
and $\phi$, two CP violation observables 
$A_{CP} = \frac{\alpha + \bar{\alpha}}{\alpha - \bar{\alpha}}$
and 
$\phi_{CP} = \frac{\phi + \bar{\phi}}{2}$ can be defined. 
If CP conservation holds, $A_{CP}$ and $\phi_{CP}$ will be strictly equal to 0.
Any nonzero value of $A_{CP}$ and $\phi_{CP}$ indicates the CP violation in
hyperon decay.

In this work, we will focus on the following two decay channel,
\begin{subequations}
	\label{eq:xixi}
	\begin{align}
		e^+e^- &\to J/\psi \to \Xi^-\bar{\Xi}^+ \to \Lambda(\to n\pi^0)\pi^- \bar{\Lambda} (\to \bar{p}^-\pi^+)\pi^+, (\text{neutron channel}) \label{eq:neutronchannel}\\
		e^+e^- &\to J/\psi \to \Xi^-\bar{\Xi}^+ \to \Lambda(\to p^+\pi^-)\pi^- \bar{\Lambda} (\to \bar{n}\pi^0)\pi^+. (\text{anti-neutron channel}) \label{eq:antineutronchannel}
	\end{align}
\end{subequations}
Figure~\ref{fig:xixidig} shows the diagram of the decay channels.
The hadronic decays of two hyperons $\Xi^-$ (S = 2) and $\Lambda$ (S = 1) will be
studied. There are total 10 parameters in this process, two production parameter
$\alpha_{J/\psi}$ and $\Delta \Phi$, four decay asymmetry parameters of $\Xi^-$,
$\alpha^{\Xi}$, $\phi^{\Xi}$, $\bar{\alpha}^{\Xi}$, and $\bar{\phi}^{\Xi}$,
four decay asymmetry parameters of $\Lambda$, $\alpha^{\Lambda}_{-}$, 
$\bar{\alpha}^{\Lambda}_{+}$, $\alpha^{\Lambda}_{0}$, and $\bar{\alpha}^{\Lambda}_{0}$.
The unique advantage of this work is that we can measure the four decay modes of 
$\Lambda$ decay simultaneously,
\begin{equation}
	\begin{aligned}
		\Lambda &\to p\pi^-,\\
		\Lambda &\to n\pi^0,\\
		\bar{\Lambda} &\to \bar{p}\pi^+,\\
		\bar{\Lambda} &\to \bar{n}\pi^0,
	\end{aligned}
\end{equation}
so that, the isospin amplitude can be studied by combining the four decay modes. 
For $\Lambda \to p \pi^-$, the S-wave and P-wave can be expressed as
\begin{equation}
	\begin{aligned}
		S(\Lambda_-) = -\sqrt{\frac{2}{3}} S_{11} e^{i(\delta_{11}^S + \phi_1^S)} 
		+ \sqrt{\frac{1}{3}} S_{33} e^{i(\delta_{33}^S + \phi_3^S)},\\
		P(\Lambda_-) = -\sqrt{\frac{2}{3}} P_{11} e^{i(\delta_{11}^P + \phi_1^P)} 
		+ \sqrt{\frac{1}{3}} P_{33} e^{i(\delta_{33}^P + \phi_3^P)},
	\end{aligned}
\end{equation}
for $\Lambda \to n \pi^0$,
\begin{equation}
	\begin{aligned}
		S(\Lambda_0) = \sqrt{\frac{1}{3}} S_{11} e^{i(\delta_{11}^S + \phi_1^S)} 
		+ \sqrt{\frac{2}{3}} S_{33} e^{i(\delta_{33}^S + \phi_3^S)},\\
		P(\Lambda_0) = \sqrt{\frac{1}{3}} P_{11} e^{i(\delta_{11}^P + \phi_1^P)} 
		+ \sqrt{\frac{2}{3}} P_{33} e^{i(\delta_{33}^P + \phi_3^P)},
	\end{aligned}
\end{equation}
where $S$ and $P$ is isospin amplitudes with subscript convention 
$S_{2\Delta I, 2I}$ and $P_{2\Delta I, 2I}$. The average of $\alpha^{\Lambda}$ 
for two modes is the same as the values in the $|\Delta I| = 1/2$ limit,
\begin{equation}
	\alpha^{\Lambda} := \frac{2\alpha^{\Lambda}_{-} + \alpha^{\Lambda}_{0}}{3} = 2 S_{11} P_{11} \cos(\phi_{1}^P - \phi_1^S).
\end{equation}
The first order correction of $|\Delta I| = 3/2$ is given as:
\begin{equation}
	\begin{aligned}
		\frac{\alpha_{-}^{\Lambda}-\alpha_{0}^{\Lambda}}{\alpha^{\Lambda}}&=\frac{3}{\sqrt{2}} \frac{\Delta \alpha_{3 / 2}}{\cos \left(\phi_{1}^{P}-\phi_{1}^{S}\right)}+3\left(2 s^{2}-1\right) \Delta_{\Lambda}, \\
\Delta \alpha_{3 / 2} &=p_{3}\left[\left(1-s^{2}\right) \cos \left(2 \phi_{1}^{P}-\phi_{1}^{S}-\phi_{3}^{P}\right)-s^{2} \cos \left(\phi_{1}^{S}-\phi_{3}^{P}\right)\right] \\ 
		&+s_{3}\left[s^{2} \cos \left(2 \phi_{1}^{S}-\phi_{1}^{P}-\phi_{3}^{S}\right)-\left(1-s^{2}\right) \cos \left(\phi_{1}^{P}-\phi_{3}^{S}\right)\right] 
	\end{aligned}
\end{equation}
where $s:= S_{11}$, $s_3 := S_{33}/S_{11}$, and $p_3 := P_{33}/P_{11}$. 
We can construct isospin averages of the observables from two isospin modes
to recover the results in the $|\Delta I| = 1/2$ limit and require a better 
precision,
\begin{equation}
	A^{\Lambda}_{CP} := \frac{2 A^-_{CP} + A^0_{CP}}{3} = - \tan (\delta^P_{11} - \delta^S_{11}) \tan (\phi^P_1 - \phi^S_1).
\end{equation}
The CP observables $A^{\Xi}_{CP}$ and $\phi_{CP}^{\Xi}$ of $\Xi^-$ can also be measured to have a cross check with
the charged channel measurement.

\begin{figure}
	\unitlength = 1mm
	\centering
	\begin{fmffile}{xixi}
		\begin{fmfgraph*}(120,75)
			\fmfleft{i1,i2}
			\fmfright{o1,o2,o3,o4,o5,o6}
			\fmflabel{$e^-$}{i1}
			\fmflabel{$e^+$}{i2}
			\fmflabel{$n(p^+)$}{o6}
			\fmflabel{$\bar{p}^-(\bar{n})$}{o1}
			\fmflabel{$\pi^0(\pi^-)$}{o5}
			\fmflabel{$\pi^-$}{o4}
			\fmflabel{$\pi^+$}{o3}
			\fmflabel{$\pi^+(\pi^0)$}{o2}
			%\fmflabel{$i\sqrt{\alpha}$}{v1}
			%\fmflabel{$i\sqrt{\alpha}$}{v2}
			\fmfblob{.08w}{v2}
			\fmf{fermion}{i1,v1,i2}
			\fmf{double_arrow}{o1,v5}
			\fmf{double_arrow}{v6,o6}
			\fmf{vanilla}{v5,o2}
			\fmf{vanilla}{v3,o3}
			\fmf{vanilla}{v4,o4}
			\fmf{vanilla}{v6,o5}
			\fmf{double_arrow, label=$\Lambda$}{v4,v6}
			\fmf{double_arrow, label=$\Xi^-$}{v2,v4}
			\fmf{double_arrow, label=$\bar{\Xi}^+$}{v3,v2}
			\fmf{double_arrow, label=$\bar{\Lambda}$}{v5,v3}
			\fmf{photon,label=$\gamma$}{v1,v7}
			\fmf{dbl_dashes,label=$J/\psi$}{v7,v2}
		\end{fmfgraph*}
	\end{fmffile}
	\caption{Using \texttt{feynmp}}
	\label{fig:xixidig}
\end{figure}


\subsection{Previous results}

The decay process $J/\psi \to \Xi^-\bar{\Xi}^+$ belongs to family $J/\psi\to Y\bar{Y}$, where $Y$ 
stands for hyperon $\Lambda$, $\Sigma$, and $\Xi$. These decay processes are published
 or ongoing at BESIII collaboration. The results of the charged channel of $J/\psi \to \Xi^-\bar{\Xi}^+$
 and $J/\psi \to \Lambda \bar{\Lambda}$ are list in Table~\ref{table:main}.

\begin{table}[hbtp]
\centering
\normalsize
\caption[]{{\bf Summary of results.}}


  \vspace{0.2cm}
  \renewcommand{\arraystretch}{1.3}
\begin{tabular}{lll}
  \hline  \hline
   Parameter  & \multicolumn{1}{c}{BESIII result} & \multicolumn{1}{c}{Previous result} \\
  \hline
  $\alpha_{\psi}$           & $\phantom{-}0.586\pm 0.012 \pm0.010$ &$\phantom{-}0.58\pm0.04 \pm 0.08$ \hfill \cite{Ablikim:2016iym}\\
  $\Delta\Phi$              & $\phantom{-}1.213\pm 0.046 \pm 0.016$~rad &\multicolumn{1}{c}{--} \\

  $\alpha_{\Xi}$           & $-0.376\pm0.007\pm0.003$&$-0.401\pm0.010$\hfill \cite{Zyla:2020zbs}\\
    $\phi_{\Xi}$             & $\phantom{-}0.011\pm0.019\pm0.009$ rad &$-0.037\pm0.014$~rad\hfill \cite{Zyla:2020zbs}\\
  $\overline{\alpha}_{\Xi}$        & $\phantom{-}0.371\pm0.007\pm0.002$ &\multicolumn{1}{c}{--}\\

  $\overline{\phi}_{\Xi}$          & $-0.021\pm0.019\pm0.007$ rad &\multicolumn{1}{c}{--}\\
  $\alpha^{\Lambda}_-$           & $\phantom{-}0.757 \pm 0.011 \pm 0.008$ &$\phantom{-}0.750 \pm 0.009\pm 0.004$\hfill \cite{Ablikim:2018zay}\\
  $\overline{\alpha}^{\Lambda}_+$          & $-0.763 \pm 0.011 \pm 0.007$ &$-0.758\pm 0.010 \pm 0.007$\hfill\cite{Ablikim:2018zay}\\
  $\alpha^{\Lambda}_0$           & \multicolumn{1}{c}{--} & $\phantom{-}0.74\pm0.05$\\
  $\overline{\alpha}^{\Lambda}_0$          & $-0.692 \pm 0.016 \pm 0.006$ & \multicolumn{1}{c}{--}\\
    \hline  \hline
    $\xi_P - \xi_S$      & $\phantom{-}(1.2\pm3.4\pm0.8)\times10^{-2}$~rad & \multicolumn{1}{c}{--} \\
    $\delta_P - \delta_S$  & $(-4.0\pm3.3\pm1.7)\times10^{-2}$~rad & $\phantom{-}(10.2\pm3.9)\times10^{-2}$ ~rad\hfill\cite{Huang:2004jp} \\
  \hline  \hline
  $A_{\rm CP}^{\Xi}$ & $\phantom{-}(6.0\pm13.4\pm5.6)\times10^{-3}$ & \multicolumn{1}{c}{--} \\
  $\Delta\phi_{\rm CP}^{\Xi}$ & $(-4.8\pm13.7\pm2.9)\times10^{-3}$ ~rad & \multicolumn{1}{c}{--} \\
  $A_{\rm CP}^{\Lambda}$ & $(-3.7\pm11.7\pm9.0)\times10^{-3}$  &$(-6\pm12\pm7)\times10^{-3}$ \hfill\cite{Ablikim:2018zay} \\
  \hline  \hline
  $\left<\phi_{\Xi}\right>$            & $\phantom{-}0.016 \pm 0.014 \pm 0.007$~rad & \\
  \hline  \hline
\end{tabular}
  \vspace{0.3cm}
 \label{table:main}
\end{table}


