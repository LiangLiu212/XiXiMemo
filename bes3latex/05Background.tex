\section{Background estimation}
\label{sec:background}
\subsection{Inclusive MC}
The potential backgrounds from other decay processes that might be present
in the data are studied by analyzing the official inclusive MC sample.
After final events selection, a topology method is used to classify the 
survived events. Table.~\ref{tab:topopm2018}~\ref{tab:topopp2018} list the 
dominantly contributing processes
for two decay channels and four data sets. Three variables $M(\Xi^-)$,
$M(\bar{\Xi}^+)$ and $M(n/\bar{n})$ are used to separate the signal 
and background. If a background channel has very similar shapes in these 
three mass spectra with the signal shapes, it is called a peaking background.
They can be categorized into three groups: charged decay channel, 
peaking background, and non-peaking background.



\begin{table}[hbtp]
	\centering
	\normalsize
	\caption[]{}
	\label{tab:topopm2018}
	\begin{tabular}{clcccc}
		\hline \hline
		No. & Decay tree & 2009 & 2012 & 2018 & 2019 \\
		\hline
		1 &  $
		J/\psi \rightarrow \eta_{c} \gamma ,
		\eta_{c} \rightarrow \bar{\Xi}^{+} \Xi^{-} ,
		\bar{\Xi}^{+} \rightarrow \pi^{+} \bar{\Lambda} ,
		\Xi^{-} \rightarrow \pi^{-} \Lambda ,
		\bar{\Lambda} \rightarrow \pi^{+} \bar{p} ,
		$ & 28 & 178 & 611 &535  \\
		2 & $
		J/\psi \rightarrow \pi^{+} \pi^{-} \Lambda \bar{\Lambda} ,
		\Lambda \rightarrow \pi^{0} n ,
		\bar{\Lambda} \rightarrow \pi^{+} \bar{p}
		$ & 23 & 108 & 429 & 348 \\
		3 & $
		J/\psi \rightarrow \bar{\Xi}^{+} \Xi^{-} ,
		\bar{\Xi}^{+} \rightarrow \pi^{+} \bar{\Lambda} ,
		\Xi^{-} \rightarrow \pi^{-} \Lambda ,
		\bar{\Lambda} \rightarrow \pi^{+} \bar{p} ,
		\Lambda \rightarrow \pi^{-} p
		$ & 4 & 21 & 86 & 113 \\
		4 & Others & 20 & 197 & 434 & 340 \\
		\hline \hline
	\end{tabular}
\end{table}

\begin{table}[hbtp]
	\centering
	\normalsize
	\caption[]{}
	\label{tab:topopp2018}
	\begin{tabular}{clcccc}
		\hline \hline
		No. & Decay tree & 2009 & 2012 & 2018 & 2019 \\
		\hline
		1 & $
		J/\psi \rightarrow \eta_{c} \gamma ,
		\eta_{c} \rightarrow \bar{\Xi}^{+} \Xi^{-} ,
		\bar{\Xi}^{+} \rightarrow \pi^{+} \bar{\Lambda} ,
		\Xi^{-} \rightarrow \pi^{-} \Lambda ,
		\bar{\Lambda} \rightarrow \pi^{0} \bar{n} ,
		$  & 30 & 175 & 612 &345  \\
		2 & $
		J/\psi \rightarrow \pi^{+} \pi^{-} \Lambda \bar{\Lambda} ,
		\Lambda \rightarrow \pi^{-} p ,
		\bar{\Lambda} \rightarrow \pi^{0} \bar{n}
		$ & 17 & 114 & 373 & 253 \\
		3 & $
		J/\psi \rightarrow \bar{\Xi}^{+} \Xi^{-} ,
		\bar{\Xi}^{+} \rightarrow \pi^{+} \bar{\Lambda} ,
		\Xi^{-} \rightarrow \pi^{-} \Lambda ,
		\bar{\Lambda} \rightarrow \pi^{+} \bar{p} ,
		\Lambda \rightarrow \pi^{-} p
		$ & 6 & 20 & 200 & 149 \\
		4 & Others & 18 & 123 & 431 & 235 \\
		\hline \hline
	\end{tabular}
\end{table}



The charged decay channel has been well studied. The mDIY MC samples for 
charged decay channel are generated according to \cite{}. The mass spectra
of $M(\Xi^-)$, $M(\bar{\Xi}^+)$ and $M(n/\bar{n})$ for this channel are 
ploted in Fig.~\ref{fig:bg1mass}. Figure~\ref{fig:bg1MvM} shows the 2D 
mass distribution $M(\Xi^-)$ v.s. $M(\bar{\Xi}^+)$.
This background will be subtracted in the maximum 
log likelihood fit. 

\begin{figure*}[hp]
	\centering
	\mbox
	{
		\begin{overpic}[width=1.0\textwidth]{Figure/bg1mass.eps}
		\end{overpic}
	}
	\caption{}
	\label{fig:bg1mass}
\end{figure*}

\begin{figure*}[hp]
	\centering
	\mbox
	{
		\begin{overpic}[width=0.8\textwidth]{Figure/bg1MvM.eps}
		\end{overpic}
	}
	\caption{}
	\label{fig:bg1MvM}
\end{figure*}


The exclusive decay $J/\psi \to \eta_c\gamma \to\Xi^- \bar{\Xi}^+\gamma$
is a peaking background. According to PDG, the related  branch fractions 
are $\mathcal{B}(J/\psi \to \eta_c\gamma) = (17\pm0.4)\%$ and
$\mathcal{B}(\eta_c \to \Xi^- \bar{\Xi}^+) = (9.0\pm2.6)\times10^4$.
The process $e^-e^+ \to J/\psi \to \eta_c\gamma$ has a angular distribution
$1+\cos^2\theta$. The spin density matrix for $\eta_c \to \Xi^- \bar{\Xi}^+$
is diag(1, -1, 1, 1). The decay matrix for the hyperons have been discussed 
in Sec.~\ref{}. A mDIY MC sample with 30 number of events are 
generated to estimate the background in the data.

For the phase space decay $J/\psi \to \pi^+\pi^- \Lambda \bar{\Lambda}$
and the rest decay channels, figure~\ref{fig:bg3mass}~\ref{fig:bg3MvM} 
show the distributions
of variables $M(\Xi^-)$, $M(\bar{\Xi}^+)$ and $M(n/\bar{n})$. Since the 
shapes for $M(\Xi^-)$ and $M(\bar{\Xi}^+)$ are continuous and flat, a 
sideband method will be used to estimate this background.

\begin{figure*}[hp]
	\centering
	\mbox
	{
		\begin{overpic}[width=1.0\textwidth]{Figure/bg3mass.eps}
		\end{overpic}
	}
	\mbox
	{
		\begin{overpic}[width=1.0\textwidth]{Figure/bg4mass.eps}
		\end{overpic}
	}
	\caption{}
	\label{fig:bg3mass}
\end{figure*}

\begin{figure*}[hp]
	\centering
	\mbox
	{
		\begin{overpic}[width=0.45\textwidth]{Figure/bg3MvM.eps}
		\end{overpic}
	}
	\mbox
	{
		\begin{overpic}[width=0.45\textwidth]{Figure/bg4MvM.eps}
		\end{overpic}
	}
	\caption{}
	\label{fig:bg3MvM}
\end{figure*}

\subsection{Miscombination background}

When the $\pi^0$ are reconstructed from the EMC photon shower queues, there 
is a probability that one or both of the photon showers are noise or fake
photons. That event might survive the final event selection. Althought it is a 
real signal event, the miscombination of photon will lead to a false four
momenta of $\pi^0$ and a wrong angular reconstruction of neutron or anti-neutron.
A photon matched angle is used to judge if the reconstructed photon is
correct or not, which is defined as following: there are two photons in the MC 
truth, $\gamma^1_{\rm true}$ and $\gamma^2_{\rm true}$, and two reconstructed
photons $\gamma^1_{\rm rec}$ and $\gamma^2_{\rm rec}$. If $\theta(\gamma^1_{\rm true},
\gamma^1_{\rm rec}) + \theta( \gamma^2_{\rm true} , \gamma^2_{\rm rec}) < 
\theta(\gamma^1_{\rm true}, \gamma^2_{\rm rec}) + \theta( \gamma^2_{\rm true} , 
\gamma^1_{\rm rec})$, then, the matched angle for photon 1 and photon 2 is 
$\theta^1_{\rm match} = \theta(\gamma^1_{\rm true}, \gamma^1_{\rm rec})$ and
$\theta^2_{\rm match} = \theta(\gamma^2_{\rm true}, \gamma^2_{\rm rec})$, 
respectively; else 
$\theta^1_{\rm match} = \theta(\gamma^1_{\rm true}, \gamma^2_{\rm rec})$,
$\theta^2_{\rm match} = \theta(\gamma^2_{\rm true}, \gamma^1_{\rm rec})$.
Figure~\ref{fig:gamangle} show the distributions of the photon matched angle.
It's obvious that the matched angle for some photons are too large to be 
correct. At the stage of MC study, requirements on photon matched angle
$\theta^1_{\rm match} < 0.3$~rad and $\theta^2_{\rm match} < 0.3$~rad can
be applied to separate the signal and mis-combination background. As shown
in Fig.~\ref{fig:recmass}, the shape of mis-combination background are relatively 
flat compared to one of the signal. It can be used as a variable to distinguish
between signal and background. It is also expectable that the shapes of 
signal and background are very similar for $M(\Xi^-)$, $M(\bar{\Xi}^+)$
and $M(\Lambda)$.
\begin{figure*}[hp]
	\centering
	\mbox
	{
		\begin{overpic}[width=0.9\textwidth]{Figure/gamangle.eps}
		\end{overpic}
	}
	\caption{}
	\label{fig:gamangle}
\end{figure*}
\begin{figure*}[hp]
	\centering
	\mbox
	{
		\begin{overpic}[width=0.9\textwidth]{Figure/recmass.eps}
		\end{overpic}
	}
	\caption{}
	\label{fig:recmass}
\end{figure*}
\begin{figure*}[hp]
	\centering
	\mbox
	{
		\begin{overpic}[width=0.9\textwidth]{Figure/baryonangle.eps}
		\end{overpic}
	}
	\caption{}
	\label{fig:baryonangle}
\end{figure*}

Figure~\ref{fig:baryonangle} show that the mis-combination background will 
lead to a worse resolution of the reconstruction of neutron/anti-neutron 
position. A large mDIY MC sample will be used to estimate the mis-combination
background.
