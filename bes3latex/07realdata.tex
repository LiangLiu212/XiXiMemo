\section{Real data}
\subsection{Blind analysis}
In order to reduce the experimenter's bias, this measurement
adopts a blind analysis technique~\cite{Roodman:2003rw}. As discussed in 
Sec.~\ref{sec:mcsimulation} and Sec.~\ref{sec:background},
the signal MC and background MC are generated according to the
knowledge that we have learned from previous experiments. The events
selections in Sec.~\ref{sec:eventselection} are optimized according 
to the MC simulation without looking at the data. The total measurement
procedure is fixed on a sub-sample of the data, 2009 + 2012 + 1/3 of 2018 
and 2019 data samples.
The systematic impact of the defferences between MC and data are
studied by applying a hidden answer method. Once all the strategies
are sattled, it will be performed to the full data sets to open the
hidden box. The follows are the comparisons of distribution  
between data and MC which show the agreement of them.

\subsection{Signal yield}
According to the discussion in Sec.~\ref{sec:background}, the invariant mass
of neutron/anti-neutron is chosen as a variable to obtain the signal yield,
with an unbinned maximum likelihood fit. The signal shape is modeled as the
likelihood function by a tool {\sc RooKeysPdf} in {\sc RooFit} to describe  
the signal. As shown in Fig.~\ref{fig:bkg2012xixipm}, the 
shape of the miscombination background drops rapidly from 0.96 to 0.98~GeV/$c^2$.
A standard argus function is not enough to describe it. We choose the product
of an argus function and a 3rd order polynomial to model this background shape.

In the fit of the invariant mass of neutron/anti-neutron, the parameters for 
the background function is fixed by fitting the background shape which is 
from mDIY MC. The blue line in Fig.~\ref{fig:bkg2012xixipm} is the 
parametric background shape. The full likelihood function is the sum of 
background function and signal shape convolved with a Gaussian 
function. The results of fit is show in Fig~\ref{fig:bkg2012xixipm}. 
The background have to 
be subtrack from this fit results. The signal yield is calculated as follow 
equation:
\begin{equation}
	N_{\rm sig} = N_{\rm fit} - N_{\rm sideband} - N_{\rm peaking},
\end{equation}

where $N_{\rm sideband}$ is the background events estimated by a sideband method 
which is defined as shown in Fig.~\ref{fig:xi2dmassdata}. $N_{\rm peaking}$
is the number of events of peaking background channel $J/\psi \to \gamma \eta_c$,
which is estimated with a exclusive MC simulation.


\begin{figure*}[hp]
	\centering
		\mbox
		{
			\begin{overpic}[width=0.45\textwidth]{Figure/2012xixipmv5/modelbkg.eps}
			\end{overpic}
		}
		\mbox
		{
			\begin{overpic}[width=0.45\textwidth]{Figure/2012xixipmv5/massfit.eps}
			\end{overpic}
		}
		\caption{2012 neutron channel}
		\label{fig:bkg2012xixipm}
\end{figure*}


\begin{figure*}[hp]
	\centering
		\mbox
		{
			\begin{overpic}[width=0.45\textwidth]{Figure/2012xixipmv5/xi2dmassdata.eps}
			\end{overpic}
		}
		\caption{2012 neutron channel}
		\label{fig:xi2dmassdata}
\end{figure*}



\subsection{Variables comparison}
The agreement between data and MC of the varibles which are used as 
event selection requirements to suppress the background are necessary
to be checked. Figure~\ref{fig:variables2012xixipm} take the neutron 
channel in 2012 as an example to show that distributions in a order
1) $\bar{\Lambda}$ decay length, 2) $\bar{\Xi}^+$ decay length, 
3) $\bar{\Xi}^+$ invariant mass, 4) $\Xi^-$ invariant mass,
5) $\bar{\Lambda}$ invariant mass, 6) $\cos \theta (\bar{\Xi}^+)$,
7) $n$ invariant mass, 8) the angle between $\Lambda$ and $\gamma_1$,
9) the angle between $\Lambda$ and $\gamma_2$, 
10) $\chi^2$ of secondary vertex fit of $\bar{\Xi}^+$,
11) $\chi^2$ of secondary vertex fit of $\bar{\Lambda}$,
12) $\chi^2$ of kinematic fit.
The distributions of other data samples are very similar to 
the example and they all have a good agreement between data 
and corresponding MC simualtion. We will leave the rests in 
Appendix~\ref{app:variablescomparison}.


\begin{figure*}[hp]
	\centering
	\foreach \n in {1, 2, 3, 4, 5, 6, 7, 8, 9, 10, 11, 12}{
		\mbox
		{
			\begin{overpic}[width=0.31\textwidth]{Figure/2012xixipmv5/can\n.eps}
			\end{overpic}
		}
		}
		\caption{2012 neutron channel}
		\label{fig:variables2012xixipm}
\end{figure*}

\subsection{Transverse momentum and polar angle}
The transverse momenta and polar angle of the resonance
$\Xi$ and $\Lambda$ and the final states proton, neutron,
charge and neutral pion are also ploted to show that 
there is no bias. Figure~\ref{fig:pt2012xixipm} and 
\ref{fig:costheta2012xixipm} present the distributions of
transverse momenta and polar angle for the neutron channel in 2012, respectively.
All the particles in decay chain are listed in order 
1) $\bar{\Xi}^+$, 2) $\Xi^+$, 3) $\Lambda$, 4) $\bar{\Lambda}$,
5) proton, 6) neutron,
7) $\pi^+$ from $\bar{\Xi}^+$, 8) $\pi^+$ from $\bar{\Lambda}$,
9) $\pi^-$ from $\Xi^-$, 10) $\pi^0$, 11) $\gamma_1$, 12) $\gamma_2$.
It has to be mentioned that we plot the distributions of the momentum 
of $\pi^0$ and the energy of $\gamma$ considering of the 
independent freedem of detection. The distributions of the rest 
data samples can be found in Appendix~\ref{app:transverse}.


\begin{figure*}[hp]
	\centering
	\foreach \n in {13, 14, 15, 16, 17, 18, 19, 20, 21, 22, 23, 24}{
		\mbox
		{
			\begin{overpic}[width=0.31\textwidth]{Figure/2012xixipmv5/can\n.eps}
			\end{overpic}
		}
		}
		\caption{2012 neutron channel}
		\label{fig:pt2012xixipm}
\end{figure*}




\begin{figure*}[hp]
	\centering
	\foreach \n in {25, 26, 27, 28, 29, 30, 31, 32, 33, 34, 35, 36}{
		\mbox
		{
			\begin{overpic}[width=0.31\textwidth]{Figure/2012xixipmv5/can\n.eps}
			\end{overpic}
		}
		}
		\caption{2012 neutron channel}
		\label{fig:costheta2012xixipm}
\end{figure*}

\subsection{Polarization and entanglement}
\subsection{Data fit results}

