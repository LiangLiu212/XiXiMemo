\section{Formalism}
\label{sec:formalism}
Consider the production of the spin-$1/2$ hyperon-antihyperon pair $e^+e^- \rightarrow Y\overline{Y}$. The formalism presented here is derived assuming one-photon exchange and follows the modular approach, described in detail in Ref.~\cite{Perotti:2018wxm}. The spin projections of the produced $Y\bar{Y}$ are entangled and for spin-1/2 baryons the system is described by only two complex parameters - the hadronic form factors $G_E^{\psi}$ and $G_M^{\psi}$~\cite{GF16+}, just as in the case of electromagnetic production. In fact, since the global complex phase can not be observed, instead of two complex form factors 
three real parameters are commonly used: an overall factor corresponding to the modulus of the
hadronic magnetic form factor, the modulus of the form factor ratio $\rho$, and their relative phase 
$\Delta\Phi=\Phi_E^\Psi-\Phi_M^\Psi$. A non-zero phase between the form factors means that the produced particles are polarized, even if the colliding beams are unpolarized~\cite{AZ96}. The polarization is perpendicular to the reaction plane and given by a simple function of the scattering angle. It can be determined from the asymmetries of the hyperon two-body decays.

\subsection{Asymmetry decay parameters}
For a hyperon of spin 1/2 which subsequently decays weakly to a baryon of spin 1/2 and a pseudoscalar spin-0 meson, e.g. $\lam \rightarrow p \pi^{-}$, the conservation of total angular momentum requires that the final states are either in relative angular momentum 0 or 1, denoted $s$ or $p$ wave, respectively. The decay distributions are described by the asymmetry parameter $\alpha$ of the baryon angular distribution, and the baryon transverse polarization using parameters $\beta$ and $\gamma$~\cite{MS08}. The decay parameters are unique to each particular decay mode. Experimentally $\alpha$ is measured by the differential counting rate $dN/d\Omega \propto(1+\alpha \vec{P_{Y}} \cdot \vec{p}_{B})$, where $\vec{P_Y}$ is the hyperon polarization and $\vec{p}_{B}$ is the decay baryon momentum unit vector in the hyperon rest frame. Thus the $\alpha$-parameter describes the asymmetry in the angular distribution of the emitted baryon in the hyperon rest frame. The angular asymmetry also gives information about the polarization of the parent hyperon. In terms of the $s$ and $p$ amplitudes, 
\begin{equation}
    \alpha = 2Re(A_s^*\cdot A_{p} ), \ \ \beta=2Im(A_s^*\cdot A_{p}) \ \ \gamma=|A_{s}|^{2}-|A_p |^2.
\end{equation}
Experimentally $\beta$ and $\gamma$ are reconstructed from the measured observable $\phi_{Y}$,
\begin{equation}
    \label{betagamma}
    \beta=\sqrt{1-\alpha^{2}}\sin\phi_{Y}, \ \gamma = \sqrt{1-\alpha^2}\cos\phi_{Y}.
\end{equation}
If baryon-antibaryon symmetry holds, then the hyperon asymmetry parameter is equal and opposite to the one for the CP-odd anti-hyperon, $\alpha_{Y}=-\alpha_{\bar{Y}}$. The corresponding CP violating observable is defined as 
\begin{equation}
    \label{alpha}
    A_{\rm CP,Y} = \frac{\alpha_{Y} + \alpha_{\bar{Y}}}{\alpha_{Y}-\alpha_{\bar{Y}}}.
\end{equation}
$A_{\rm CP, Y}$ requires final state interactions in order not to vanish. A model-independent approximation of $A_{\rm CP, Y} = -\tan(\delta_p -\delta_s)\sin(\xi_p- \xi_s)$, where $(\delta_p -\delta_s)$ is the difference between the $p$- and $s$-wave final scattering phase shifts of the baryon-pseudoscalar decay products and $(\xi_p- \xi_s)$ is the CP-violating difference between weak interaction phases in the decay~\cite{MH04}. Limits on the observability of $A_{\rm CP, Y}$ can be determined from  $\tan^{-1}\phi_{Y}=\beta_{Y}/\gamma_{Y}=\tan(\delta_p -\delta_s)$. From our previous work we have good reasons to believe that final state interactions for the weak decay $\Xi\to\Lambda\pi$ is very small as the average value of $\phi_{\Xi}$, $\left<\phi\right>_{\Xi}=0.016\pm0.014\pm0.007$ is consistent with 0. With a larger data set one may determine if it is non-zero.
Small final state interactions instead enhance the CP observable constructed from the $\beta$-parameter,  
\begin{equation}
    \label{beta}
    B_{\rm CP} = \frac{\beta_{Y} + \beta_{\bar{Y}}}{\beta_{Y}-\beta_{\bar{Y}}}=\frac{\sin\phi_{Y}+\sin\phi_{\bar{Y}}}{\sin\phi_{Y}-\sin\phi_{\bar{Y}}},
\end{equation}
and can therefore be and order of magnitudes more sensitive compared to $A_{\rm CP}$~\cite{JD85, JD86, MS08}. This is because $B_{\rm CP}=-\sin(\xi_p- \xi_s)/\tan(\delta_p -\delta_s)$.
The test on $B^_{CP}$ was proposed more than 30 years ago~\cite{JD85, JD86}. Experimentally, $B_{\rm CP}$ is not an optimal observable as it is not independent of $A_{\rm CP}$ (it contains both $\alpha_{\Xi}$ and $\phi_{\Xi}$) In addition, the denominator is very small and hence such observable would provide a null result event if the numerator significantly differs from $0$. On the other hand, what is really tested is how $\phi_{\Xi}$ relates to $\bar{\phi}_{\Xi}$. Hence, $\Delta\phi = (\phi_{\Xi} + \bar{\phi}_{\Xi})/2$ and $\left< \phi_{Y} \right> =  (\phi_{Y} - \bar{\phi}_{Y})/2$, the CP-test using the decay parameter of $\phi$ is tested via

\begin{equation}
    \label{deltaphi}
    \phi_{\rm CP} = \Delta\phi.
\end{equation}

In this work, in addition to $A_{\rm CP}$, $\phi_{\rm CP}$ is measured for the first time. 
\subsection{Strong and weak phase differences}
CP violation can only be observed if there is interference between CP even and CP odd terms in the decay amplitude. Since the decay amplitude $\Xi^- \to \Lambda \pi^-$ consists of both P-wave (parity conserving) and an S-wave (parity violating) part, the leading order contribution to the CP asymmetry $A_{CP}^{\Xi}$ can be written
    \begin{equation}
      A_{CP}^{\Xi}\sim - \tan(\delta_P - \delta_S)\sin(\xi_P - \xi_S),
      \label{eq:acpphase}
  \end{equation}
where $\delta_P-\delta_S = \beta/\alpha$ denotes the strong phase difference of the final state interaction between $\Lambda$ and $\pi^-$ from the $\xim$ decay. CP violating effects would manifest itself in a non-zero weak phase difference $\xi_P - \xi_S$  \cite{JD86}, an observable that is complementary to the kaon decay parameter $\epsilon^{'}$ 
%\cite{Christenson:1964fg, AlaviHarati:1999xp, Fanti:1999nm} 
since the latter only involves an $S$-wave. A non-zero $A_{CP}^{\Xi}$ requires both non-zero strong and weak phase contributions. Therefore,  to pinpoint the weak phase difference, an independent measurement of the strong phase is required. However, since the latter is extracted from the $\phi_\Xi$ parameter, that has been found to be small ($-0.037\pm0.014$\cite{MH04,PDG18}), CP violating signals in $A_{CP}^\Xi$ are strongly suppressed and difficult to interpret in terms of the weak phase difference.

An independent and direct CP symmetry test in $\Xi^-\to\Lambda\pi^-$ is provided by determining the  $\Delta \phi_{CP}$ value. In the leading order, it is related to the weak phase difference in the following way:
\begin{equation}
 (\xi_P - \xi_S)_{\rm{LO}} = \frac{\beta + \bar{\beta}}{\alpha - \bar{\alpha}}\approx \frac{\sqrt{1-\alpha^2}}{\alpha}\Delta \phi_{CP}.  
 \label{eq:betaprime}
\end{equation} 
This means that the sensitivity to CP violation effects is enhanced by an order of magnitude with respect to that of the $A_{CP}^{\Xi}$ observable\cite{JD85, JD86}. However, to measure 
$\Delta \phi_{CP}$ using standard techniques requires that the polarization is either precisely known or identical for $\Xi^-$ and $\overline{\Xi}^+$ in separate experiments.  In principle, this can be handled if one has access to huge data samples \cite{MS08}. However, the precision will be limited by systematic effects due to asymmetries in the production of the baryon and antibaryon. Here, we present an alternative approach where the baryon and the antibaryon are entangled CP eigenstates and analyzed simultaneously.

\subsection{Overall reference system}
\noindent We are interested in the process (\ref{e+e-toxixibar}), 
\begin{equation}
    e^{+}e^{-}\to J/\psi\rightarrow \xim\xibarp\to\lam\pi^{-}\lambar\pi^{+}\to p\pi^{-}\pi^{-}\bar{p}\pi^{+}\pi^{+}.
\end{equation}
%\begin{align}
%	\mathbf{p}_{\xim}  &=  - \mathbf{p}_{\xibarp}= \mathbf{p},  \\
%	\mathbf{k}_{e^-}  &= - \mathbf{k}_{e^+} = \mathbf{k}. 
%\end{align}

%\noindent The scattering plane is spanned by the $e^+$ and the outgoing $\xim$ where the normal is defined by $\hat{n}=\hat{\mathbf{p}}\times\hat{\mathbf{k}}$. The $\xim$ scattering angle is given by

%\begin{equation}
%    	\cos\theta_{\xim}= \hat{\mathbf{p}}\cdot \hat{\mathbf{k}}.
%	\end{equation}

%\begin{figure}[h]
%\begin{center}
%\includegraphics[width=1.\linewidth]{fig/formalism/process.png}
%\caption{The coordinate system of the $e^+e^-\to J/\psi\to\xim\xibarp$ process.}
%\label{fig:coordinatesystem}
%\end{center}
%\end{figure}

\noindent To describe the full differential cross section of the exclusively measured process the nine angles given in vector $\xi$ are needed, $\xi= ( \theta_{\xim}, \theta_{\lam}, \varphi_{\lam}, \theta_{\bar{\lam}}, \varphi_{\bar{\lam}}, \theta_{p}, \varphi_{p}, \theta_{\bar{p}}, \varphi_{\bar{p}})$. Besides $\theta_{\xim}$, the other eight observables are the polar and azimuthal helicity angles of the $\Lambda$, $\bar{\lam}$, proton and anti-proton, respectively~\cite{MA10, BAM_licui}. With the helicity angles one can estimate the eight unknown parameters of the process, $\Omega = (\alpha_{\psi}, \ \Delta\Phi, \ \alpha_{\lam}, \ \alpha_{\lambar}, \ \alpha_{\xim}, \ \alpha_{\xibarp}, \ \phi_{\Xi}, \ \bar{\phi}_{\xibarp})$. 

%From a right-handed coordinate system with the basis vectors
%\begin{equation}
%	\hat{\textbf{e}}_{x} = \frac{1}{\sin{\theta}}(\hat{\textbf{p}} \times \hat{\textbf{k}})\times \hat{\textbf{p}} , \ \
%	\hat{\textbf{e}}_{y} = \frac{1}{\sin{\theta}}(\hat{\textbf{p}} \times \hat{\textbf{k}}), \ \
%	\hat{\textbf{e}}_{z} = \hat{\textbf{p}},
%\end{equation}
%the helicity angles are obtained. As seen from the basis vector definitions the helicity angles are defined with respect to the direction of the outgoing hyperon in the CM frame ($\hat{e}_{z}$). 

%\noindent The density matrix of the produced $\Xi\bar\Xi$ system is given by \cite{Perotti:2018wxm}:
%\begin{equation}
%    \rho_{\Xi\bar \Xi}\propto \sum_{\mu,\bar\nu=0}^3C_{\mu\bar\nu}\sigma_\mu^\Xi\otimes \sigma_{\bar\nu}^{\bar\Xi},
%\end{equation}
%where the production matrix is given by Eqn.~(23) in  \cite{Perotti:2018wxm}: $C_{\mu\bar\nu}\to C_{\mu\bar\nu}(\theta_\Xi;\alpha_{J/\psi},\Delta\Phi)$.
%Using replacement Eqn.~(42) and finally trace for the unmeasured polarization 
%of the protons one obtains the final differential distribution in the form: 
%\begin{equation}
%    \frac{d\Gamma}{d\xi}\propto \sum_{\mu,\bar\nu=0}^3\sum_{\mu'=0}^3\sum_{\bar\nu'=0}^3C_{\mu\bar\nu}\ 
%    a_{\mu\mu'}^\Xi\ a_{\mu'0}^\Lambda\ 
%    a_{\bar\nu\bar\nu'}^{\bar\Xi}\ a_{\bar\nu'0}^{\bar\Lambda}
%\end{equation}
%where the $a_{\mu\nu}$ matrices for $1/2\to 1/2+0$ decays are given by Eqn.~(50) in Ref.~\cite{Perotti:2018wxm}. More explicitly the matrices are the function of the following 
%helicity variables and decay parameters: $a_{\mu\mu'}^\Xi\to a_{\mu\mu'}^\Xi(\theta_\Lambda,\phi_\Lambda; \alpha_{\Xi},\phi_\Xi)$,
% $a_{\bar\nu\bar\nu'}^{\bar\Xi}\to a_{\bar\nu\bar\nu'}^{\bar\Xi}(\theta_{\bar\Lambda},\phi_{\bar\Lambda}; \alpha_{\bar\Xi},\phi_{\bar\Xi})$,
% $a_{\mu'0}^\Lambda\to a_{\mu'0}^\Lambda(\theta_p,\phi_p; \alpha_{\Lambda})$ and $a_{\bar\nu'0}^{\bar\Lambda}\to a_{\bar\nu'0}^{\bar\Lambda}(\theta_{\bar p},\phi_{\bar p}; \alpha_{\bar\Lambda})$.
 
\subsection{Phenomenology description of \texorpdfstring{$e^+e^-\to J/\psi,\psi(2S)\to \Xi\bar\Xi$}{eeXiXi}}\label{sub:C}

 Suppose one has produced a ``mother'' particle that
decays further. One wants to change from the production frame of this state to its rest frame. 
Given the state's three-momentum 
\begin{equation}
  {\bf p}_m = 
p_m \, (\cos\varphi_m\sin\theta_m,\sin\varphi_m\sin\theta_m,\cos\theta_m)
\end{equation}
and the $z$-axis in the production frame, one possibility would be to 
perform a single rotation that aligns ${\bf p}_m$ with the $z$-axis. 
Subsequently one then boosts to the rest frame of the mother particle. 
The single rotation would be around an axis perpendicular to 
${\bf p}_m$ and $\hat {\bf z}$. Yet when viewed as rotations around 
the coordinate axes this amounts to a succession of three rotations. 
Viewed as active rotations these are 
(a) a rotation around the $z$-axis by $-\varphi_m$; (b) a rotation around 
the $y$-axis by $-\theta_m$; (c) a rotation around the $z$-axis by $+\varphi_m$; see also Ref.~\cite{Jacob:1959at}. 
In principle, however, the first two rotations are sufficient to align
${\bf p}_m$ with the $z$-axis. In line with the present BESIII analyses 
we follow this two-rotation procedure
in the present work. The rotation matrix for ${\bf p}_m$ is given by
\begin{equation}
\left(
\begin{array}{rrr}
 \cos\theta_m \cos\varphi_m & \cos\theta_m \sin\varphi_m & -\sin\theta_m \\
 -\sin\varphi_m & \cos\varphi_m & 0 \\
 \cos\varphi_m \sin\theta_m & \sin\theta_m \sin\varphi_m & \cos\theta_m \\
\end{array}
\right).\label{eq:hrot}
\end{equation}
This rotation defines in a unique way the helicity reference frame 
for a daughter particle.
In an experimental analysis the  boosts and rotations in Eqn.~\eqref{eq:hrot}
are applied recursively to all decay products of
a decay chain, thus
defining a set of  helicity variables
to describe an event. 

The production process $e^+e^-\to B_1\bar B_2$, 
viewed in the CM frame, defines a 
scattering plane and therefore a coordinate system. The $z$-axis is 
chosen along the line of flight  of the incoming
positron, i.e.\ ${\bf\hat z}={\bf p}_{e^+} = (0,0,p_{\rm in})$, where $p_{\rm in}$ 
denotes the modulus of the momentum of electron and positron in the 
CM frame. The $y$-axis is chosen to be perpendicular to 
the scattering plane. One uses the direction of the baryon $B_1$ to 
define the $y$-axis: 
\begin{eqnarray}
  \label{eq:def-initial-y}
  {\bf\hat y} := \frac{{\bf p}_{e^+} \times {\bf p}_B}%
  {\vert {\bf p}_{e^+} \times {\bf p}_B \vert}  \, .
\end{eqnarray}
Finally the $x$-axis is chosen such that $x$, $y$ and $z$ adhere to the 
right-hand rule. Thus, the general coordinate system is given by

\begin{eqnarray}
  \label{eq:def-initial-xyz}
  {\bf\hat x} := {\bf\hat y}\times{\bf\hat z}, \
  {\bf\hat y} := \frac{{\bf p}_{e^+} \times {\bf p}_B}
  {\vert {\bf p}_{e^+} \times {\bf p}_B \vert}  , \ {\bf\hat z} := \frac{{\bf p}_{e^+}}{\vert {\bf p}_{e^+} \vert } .
\end{eqnarray}

Denoting the scattering angle of $B_1$ 
by $\theta_1$, all this implies that the baryon three-momentum is
${\bf p}_B = p_{\rm out} \, (\sin\theta_1,0,\cos\theta_1)$. 
Here $p_{\rm out}$ denotes the modulus of the 
momentum of baryon and antibaryon in the CM frame.
\begin{figure}
\centering
%\includegraphics[width=0.48\textwidth,bb=0 0 551 287]{axesBB.pdf}
\includegraphics[width=1.0\textwidth]{fig/formalism/axesBB.png}
\caption[]{(color online) Orientation of the axes in baryon $B_1$ and
  antibaryon $\bar B_2$ helicity frames.}
  \label{fig:axes}
\end{figure}
  
With the above definition of the CM coordinate system,
the $y$-axis of
the helicity frame of the baryon $B_1$, ${\bf\hat y}_1$ in Fig.~\ref{fig:axes}, is the same as ${\bf\hat y}$ in
Eqn.~\eqref{eq:def-initial-y}.
Therefore, for the helicity rotation matrix
Eqn.~\eqref{eq:hrot} one uses $\theta_m =\theta_1$ and $\varphi_m =0$.
Correspondingly, to transform to the helicity frame of the antibaryon $\bar B_2$
one chooses $\varphi_m=\pi$ and $\theta_m= \pi-\theta_1$. In this way 
the $y$-axis, ${\bf\hat y}_2$, is equal $-{\bf\hat y}$. The
$y$- and $z$-axes of the helicity frames of the baryon $B_1$ and the antibaryon
$\bar B_2$ have opposite directions while it is the same direction for the $x$-axis as shown in Fig.~\ref{fig:axes}. The coordinate system for the $B_1$ are given in the spherical coordinate system where the radial component is in the momentum direction of the $\xim$, $\theta$

Thus for the baryon $B_1$ helicity frame the axes for $B_1$ are defined as
\begin{eqnarray}
  \label{eq:def-B1-xyz}
  {\bf\hat x}_1 := {\bf\hat r}_1\cos\theta_{\Lambda}\sin\varphi_{\Lambda}, \
  {\bf\hat y}_1 := {\bf\hat r}_1\sin\theta_{\Lambda}\sin\varphi_{\Lambda}, \ 
  {\bf\hat z}_1 := {\bf\hat r}_1\cos\theta_{\Lambda} := \frac{{\bf p}_{\xim}}{\vert {\bf p}_{\xim} \vert }.
\end{eqnarray}


It is well known how the spin density matrices look like for a
reaction $e^+e^-\to B_1\bar{B}_2$ where both produced particles have
spin 1/2. The results were obtained using different approaches
\cite{AZ96,Czyz:2007wi,TomasiGustafsson:2005kc,Faldt:2013gka,Faldt:2016qee,Faldt:2017kgy} but here we use expressions from Ref.~\cite{Perotti:2018wxm} compatible with the introduced helicity variables.
The $\Xi^-$ has positive parity $\eta_1=1$ and the $\bar\Xi^+$
negative parity $\eta_2=-1$.  In general only two out of four possible
helicity transitions are independent. Using $\eta_1\eta_2 =-1$ for the
baryon antibaryon pair one can set $A_{1/2,1/2}=A_{-1/2,-1/2}=:\F_1$
and $A_{1/2,-1/2}=A_{-1/2,1/2}=:\F_2$.  The transition amplitude
matrix is
\begin{equation}
\left(
\begin{array}{cc}
 {\F_1} & {\F_2} \\
 {\F_2} & {\F_1} \\
\end{array}
\right)\, .
\end{equation}

The spin density matrix for a two-particle $1/2+\overline{1/2}$ system can be expressed
in terms of a set of $4\times 4 $ matrices obtained
from the outer product, $\otimes$, of $\sigma_\mu$ and ${\sigma}_{\bar\nu}$ \cite{Tabakin:1985yv}:
\begin{equation}
  \rho_{B_1,\bar B_2}=\frac{1}{4}\sum_{\mu,\bar\nu=0}^3C_{\mu\bar\nu}(\thetap\!)\,
  \sigma_\mu^{B_1}\otimes
      {\sigma}_{\bar\nu}^{\bar B_2},
\label{eqn:sig12}
\end{equation}
{where $\sigma_\mu^{B}$ with $\mu=0,1,2,3$ represent spin-$1/2$
  base  matrices
  for a  baryon $B$ in the rest frame. The $2\times 2$ matrices
 are $\sigma_0^{B}=\mathds{1}_2$, 
$\sigma_1^{B}=\sigma_x$, $\sigma_2^{B}=\sigma_y$ and $\sigma_3^{B}=\sigma_z$.
In particular the spin matrices $\sigma_\mu^{B_1}$ and
${\sigma}_{\bar\nu}^{\bar B_2}$ are given in the helicity frames of
the baryons $B_1$ and $\bar B_2$, respectively. The axes of the frames
are defined in Fig.~\ref{fig:axes} and denoted by
${\bf\hat x}_1,{\bf\hat y}_1,{\bf \hat z}_1$ and ${\bf\hat x}_2,{\bf\hat y}_2,{\bf\hat z}_2$.
  The real coefficients $C_{\mu\bar\nu}$
  are functions of the scattering angle $\thetap$ of $B_1$.
}

Suppose one is not interested in the absolute size of the cross section but only in the 
(not normalized) angular distributions. For their description we do not need all information
contained in the two complex form factors $\F_1$ and $\F_2$. Instead we can use just two real
parameters: first, $\alpha_\psi$ as defined below and, second, the relative phase between the form
factors $\Delta\Phi=\arg(\F_1/\F_2)$, i.e.\ we disregard the normalization and 
the overall phase.  More specifically without any loss of generality we take $\F_1$ as
real and set $\F_1=\sqrt{1-\alpha_\psi}/\sqrt{2}$ and
$\F_2=\sqrt{1+\alpha_\psi}\exp(-i\Delta\Phi)$.
Only 8 coefficients $C_{\mu\bar\nu}$ are non-zero and they are given by
\begin{eqnarray}
C_{00}&=&2(1+\alpha_\psi\cos^2\!\thetap\!)   \,, \nonumber\\
C_{0 2}&=&2\sqrt{1-\alpha_\psi^2}\sin\thetap\cos\thetap\sin(\Delta\Phi)   \,, \nonumber\\
C_{1 1}&=&2\sin^2\!\thetap   \,, \nonumber\\
C_{1 3}&=&2\sqrt{1-\alpha_\psi^2}\sin\thetap\cos\thetap\cos(\Delta\Phi)   \,, \nonumber\\
C_{20}&=&-C_{0 2}   \,,\label{eqn:c1212} \\
C_{2 2}&=&\alpha_\psi C_{11}   \,, \nonumber\\
C_{3 1}&=&-C_{1 3}   \,, \nonumber\\
C_{3 3}&=&-2(\alpha_\psi+\cos^2\!\thetap\!)\, .\nonumber
\end{eqnarray}

 


 
\subsection{Decay chains}
\label{sec:decaychains}


Since the joined production density matrix of Eqn.~\eqref{eqn:sig12}
is expressed as outer products of the basis matrices $\sigma_\mu$ it
is enough to know how the latter individually transform under a decay
process. This allows for modular expressions for the 
angular distributions and we follow Ref.~\cite{Perotti:2018wxm}.

Let us consider the weak decay mode of a spin-$1/2^+$ hyperon decaying
into a spin-$1/2^+$
hyperon and a pseudoscalar meson. The angular distribution is specified by two angles $\theta$ and $\phi$, which give the direction of the
final baryon in the helicity frame of the initial hyperon.  The spin
configuration of the final system is fully specified by the spin density
matrix of the final baryon, which has spin $1/2$, since
the accompanying particle is a pseudoscalar meson.
We introduce 
a $4\times 4$ matrix $a_{\mu\nu}$ which allows to express the
$\sigma_\mu$ matrices in the mother helicity frame in terms of $\sigma_\nu^d$
matrices in the daughter helicity frame:  
\begin{equation}
\sigma_\mu\to\sum_{\nu=0}^3 a_{\mu\nu}\sigma_\nu^d\, . \label{eqn:decay12p}
\end{equation}
The decay matrix $a_{\mu\nu}$ introduced above 
allows to keep track of the spin correlation between the decay 
products of the $B_1$ and $\bar B_2$ decays chains.

Below we provide the explicit expression for the decay matrices
$a_{\mu\nu}$. The transition amplitude is:
\begin{equation}
{}_{\rm d}\langle \theta, \varphi, \lambda\vert S \vert 0,0,\kappa\rangle
_{\rm m}
\propto {\cal D}^{J *}_{\kappa,\lambda}(\Omega) B_{\lambda},
\label{eqn:amphdecay}
\end{equation}
where ${\cal D}^{J }_{\kappa,\lambda}(\Omega)={\cal D}^{J }_{\kappa,\lambda}(\varphi,\theta,0)$ is Wigner D function (see remark in concerning Wigner D function definition in Ref.~\cite{Perotti:2018wxm}) and $\theta$, $\phi$
are spherical coordinates of the daugther baryon in the mother baryon helicity frame.
The coefficients  $a_{\mu\nu}$ are then obtained by multiplying the amplitude above
by its conjugate and inserting basis $\sigma$ matrices for the mother 
and the daughter baryon.
These coefficients can be rewritten  in terms of the 
decay parameters $\alpha_D$ and $\phi_D$ defined in Ref.~\cite{PDG18}. For completeness 
we first relate the helicity amplitudes to the $s$ and $p$
wave amplitudes $A_s$ and $A_p$, corresponding respectively to the 
parity violating and parity conserving transitions. 
If a hyperon of spin $J$ decays (weakly) into a hyperon of spin $S$ and a (pseudo)scalar state, then the relation between 
helicity amplitudes and canonical amplitudes is given by \cite{Jacob:1959at}
\begin{eqnarray}
  \label{eq:hel-LS}
  B_\lambda = \sum\limits_L \left(\frac{2L+1}{2J+1}\right)^{1/2} \, (L,0;S,\lambda\vert J,\lambda) \, A_L 
\end{eqnarray}
where $(s_1,m_1,s_2,m_2\vert s, m)$ is a Clebsch-Gordan coefficient.
For $J=S=1/2$ the helicity amplitudes are\footnote{Note that the Particle Data Group \cite{PDG18} uses $-A_p=A_p^{\rm PDG}$.} 
\begin{eqnarray}
B_{-1/2}&=&\frac{A_s+A_p}{\sqrt{2}} \,, \nonumber \\
B_{1/2}&=&\frac{A_s-A_p}{\sqrt{2}} \,.
\label{eqn:JWdec}
\end{eqnarray} 
Using the normalization $|A_s|^2+|A_p|^2=|B_{-1/2}|^2+|B_{1/2}|^2=1$,
the relation between helicity amplitudes and the decay parameters
is:
\begin{eqnarray}
\alpha_D&=&-2\Re(A_s^*A_p)=|B_{1/2}|^2-|B_{-1/2}|^2  \,, \nonumber\\
\beta_D&=&-2\Im(A_s^*A_p)=2\Im(B_{1/2}B_{-1/2}^*)  \,,\label{eqn:dparam} \\
\gamma_D&=&|A_s|^2-|A_p|^2=2\Re(B_{1/2}B_{-1/2}^*),\nonumber   
\end{eqnarray}
where $\beta_D=\sqrt{1-\alpha_D^2}\sin\phi_D$ and 
 $\gamma_D=\sqrt{1-\alpha_D^2}\cos\phi_D$.
The non-zero elements  of the decay matrix $a_{\mu\nu}(\theta,\varphi;\alpha_D,\beta_D,\gamma_D)$  are: 
\begin{eqnarray}
a_{00}&=&1 \,, \nonumber\\
a_{03}&=&\alpha_D \,, \nonumber\\
a_{10}&=&\alpha_D\cos\varphi\sin\theta \,, \nonumber\\
a_{11}&=&\gamma_D \cos\theta\cos\varphi-\beta_D \sin\varphi \,, \nonumber\\
a_{12}&=&-\beta_D \cos\theta
\cos\varphi-\gamma_D \sin\varphi \,, \nonumber\\
a_{13}&=&\sin\theta \cos\varphi \,, \nonumber\\
a_{20}&=&\alpha_D\sin\theta \sin\varphi \,, \label{eqn:matrixa}\\
a_{21}&=&\beta_D \cos\varphi+\gamma_D \cos\theta \sin\varphi \,, \nonumber\\
a_{22}&=&\gamma_D\cos\varphi-\beta_D \cos\theta \sin\varphi \,, \nonumber\\
a_{23}&=&\sin\theta \sin\varphi \,, \nonumber\\
a_{30}&=&\alpha_D\cos\theta \,, \nonumber\\
a_{31}&=&-\gamma_D\sin\theta \,, \nonumber\\
a_{32}&=&\beta_D\sin\theta \,, \nonumber\\
a_{33}&=&\cos\theta\, .\nonumber
\end{eqnarray}


Here we discuss an exclusive decay chain: $e^+e^-\to J/\psi,\psi(2S)\to\Xi\bar\Xi $ where $\Xi(\bar\Xi)$
decays weakly:
$\Xi(\bar\Xi)\to\Lambda(\bar\Lambda)\pi$ and then $\Lambda(\bar\Lambda)$ decays weakly: $\Lambda\to p\pi^-(\bar\Lambda\to \bar p\pi^+)$.
The production spin density matrix is given by Eqn.~\eqref{eqn:c1212}: $C_{\mu\bar\nu}\to C_{\mu\bar\nu}(\theta_\Xi;\alpha_{\psi},\Delta\Phi)$.
Using replacements Eqn.~\eqref{eqn:decay12p} for the sequential decays
and finally taking trace for the unmeasured polarization 
of the final proton-antiproton system
one obtains the differential distribution in the form: 
\begin{equation}
\mathrm{Tr}\rho_{p\bar{p}}\propto {\cal W}=\sum_{\mu,\bar\nu=0}^3C_{\mu\bar\nu}\sum_{\mu'=0}^3\sum_{\bar\nu'=0}^3
    a_{\mu\mu'}^\Xi\ a_{\mu'0}^\Lambda\ 
    a_{\bar\nu\bar\nu'}^{\bar\Xi}\ a_{\bar\nu'0}^{\bar\Lambda}\ ,\label{eq:XiXi1}
\end{equation}
where the $a_{\mu\nu}$ matrices for $1/2\to 1/2+0$ decays are given by Eqn.~\eqref{eqn:matrixa}. The matrices are the functions of the corresponding 
helicity variables and decay parameters: $a_{\mu\mu'}^\Xi\to a_{\mu\mu'}^\Xi(\theta_\Lambda,\varphi_\Lambda; \alpha_{\Xi},\beta_\Xi,\gamma_\Xi)$,
 $a_{\bar\nu\bar\nu'}^{\bar\Xi}\to a_{\bar\nu\bar\nu'}^{\bar\Xi}(\theta_{\bar\Lambda},\varphi_{\bar\Lambda}; \alpha_{\bar\Xi},\beta_{\bar\Xi},\gamma_{\bar\Xi})$,
$a_{\mu'0}^\Lambda\to a_{\mu'0}^\Lambda(\theta_p,\varphi_p; \alpha_{\Lambda})$ and $a_{\bar\nu'0}^{\bar\Lambda}\to a_{\bar\nu'0}^{\bar\Lambda}(\theta_{\bar p},\varphi_{\bar p}; \alpha_{\bar\Lambda})$.
Therefore Eqn.~\eqref{eq:XiXi1} could be rewritten as
\begin{equation}
  {\cal W}=\sum_{\mu,\bar\nu=0}^3C_{\mu\bar\nu}(\theta_\Xi;\alpha_{\psi},\Delta\Phi)Y_{\mu\bar\nu}(\boldsymbol{\xi};\Omega),  
\end{equation}
where $\Omega=(\alpha_{\Xi},\beta_\Xi,\gamma_\Xi,
\alpha_{\bar\Xi},\beta_{\bar\Xi},\gamma_{\bar\Xi}, \alpha_{\Lambda},
\alpha_{\bar\Lambda})$ and
$\boldsymbol{\xi}=(\theta_\Lambda,\varphi_\Lambda,\theta_{\bar\Lambda},\varphi_{\bar\Lambda},\theta_p,\varphi_p,\theta_{\bar
  p},\varphi_{\bar p})$. With the information provided above the explicit
form of the matrices $C_{\mu\bar\nu}$ (Eqn.~\eqref{eqn:c1212}) and
$a_{\mu\nu}$ (Eqn.~\eqref{eqn:matrixa}) it is straightforward to
calculate the joint angular distribution using Eqn.~\eqref{eq:XiXi1},
but the expressions are lengthy.
We find that from $4^4=256$ possible terms, 100 terms are non-zero.
The relevant $Y_{\mu\bar\nu}$ elements (compare non-zero terms
in Eqn.~\eqref{eqn:c1212} ) are: $Y_{00}$, $Y_{02}$, $Y_{11}$, $Y_{13}$, $Y_{20}$, $Y_{22}$, $Y_{31}$, $Y_{33}$. For example:
\begin{equation*}
Y_{00}=({\alpha_\Xi} {\alpha_\Lambda} \cos {\theta_\Lambda}+1) ({\alpha_{\bar\Xi}} {\alpha_{\bar\Lambda}} \cos{\theta_{\bar\Lambda}}+1)  .
\end{equation*}


