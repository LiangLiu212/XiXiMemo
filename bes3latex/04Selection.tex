\section{Event Selection}
\label{sec:eventselection}
\subsection{Quick review}

\subsection{Track level selection}
\begin{itemize}
\item {\bf Good charged track}
\begin{itemize}
\item Charged tracks detected in the MDC are required to be within a polar angle ($\theta$) range of $|\rm{cos\theta}|<0.93$, where $\theta$ is defined with respect to the $z$-axis,
which is the symmetry axis of the MDC.

\item Due to the long life time of $\Xi$ and $\Lambda$, for charged tracks in final states,
	the distance of closest approach to the interaction point (IP)
must be less than 30\,cm
along the $z$-axis, $|V_{z}|$,
and less than 10\,cm
in the transverse plane, $|V_{xy}|$.
\item The total number of charged tracks should be equal to 4. This requirement can help us 
	reducing the background caused by charged channel
		$J/\psi\to \Xi^- \bar{\Xi}^+ \to \Lambda(\to p\pi^-)\pi^- \bar{\Lambda}(\to \bar{p} \pi^+) \pi^+$.
\end{itemize}
\item {\bf Good photon selection}
\begin{itemize}
  \item Photon candidates are identified using showers in the EMC.  The deposited energy of each shower must be more than 25~MeV in the barrel region ($|\cos \theta|< 0.80$) and more than 50~MeV in the end cap region ($0.86 <|\cos \theta|< 0.92$).  

\item  
To exclude showers that originate from
charged tracks,
the angle subtended by the EMC shower and the position of the closest charged track at the EMC
must be greater than 20 degrees as measured from the IP. 
%[ORIGINAL:
%To exclude showers that originate from
%charged tracks, the angle between the position of each shower in the EMC and the closest extrapolated charged track must be greater than 10 degrees.]

\item  To suppress electronic noise and showers unrelated to the event, the difference between the EMC time and the event start time is required to be within 
[0, 700]\,ns.

\item 
	There is a additional requirement for neutron channel. To veto the shower deposited by 
	neutrons in the EMC, the openning angle between the direction of $\Lambda$ and the 
		direction of a photon shower must be greater than 15 degrees. However, 
		anti-neutron channel doesn't need such requirement. It will be discussed in the following 
		sub-section.
\end{itemize}

\item {\bf Particle identification}
	\begin{itemize}
		\item  One proton/anti-proton, two $\pi^-$/$\pi^+$, and one $\pi^+$/$\pi^-$ must be identified 
			for neutron/anti-neutron channel. According to the true information as shown in Fig.~\ref{fig:trufmomxixipm},~\ref{fig:trufmomxixipp},
			the proton and pion candidates must have momenta
			$p_{pr} > 0.32 \GeV/c$ and $p_{\pi} < 0.30 \GeV/c$, respectively.
			There is no overlap between proton and pion momenta. The comparison
			of reconstruction momenta between data and MC are shown in
			Fig.~\ref{}~\ref{}.
			The alternative would be to use particle identification methods. 
			As discussed in~\cite{}, this method is not considered viable.
	\end{itemize}

\end{itemize}
\subsection{Event level selection}
For both neutron channel and anti-neutron channel, there is two legs, a charged leg 
$\Xi^- \to \Lambda(\to p\pi^-) \pi^-$ or $c.c.$ and a neutral leg
$\bar{\Xi}^+ \to \bar{\Lambda} (\to \bar{n} \pi^0)\pi^+$ or $c.c.$.
A so called single tag double tag method are used to reconstruct the decay process 
from the available pool of charged tracks and photon showers. Single tag is used to 
reconstruct the charged leg and double tag is used to reconstruct the neutral lag.
\begin{itemize}
	\item {\bf Single tag}
		\begin{itemize}
			\item The $\Lambda$ candidate is reconstructed from proton 
				and charged pion and required to pass a primary 
				vertex fit. 
			\item The $\Xi$ candidate reconstructed from the $\Lambda$
				and the remaining pion is required to pass a primary
				and a secondary vertex fit. The secondary vertex fit
				for the $\Lambda$ is set at the decay point of $\Xi$, 
				for the formed $\Xi$ is set at the  interaction point.
			\item The charged combination is selected to minimize 
				$\Delta m_{\Xi \Lambda} = ((m_{p\pi\pi} - m_{\Xi})^2
				+ (m_{p\pi} - m_{\Lambda})^2)^{1/2}$, where $m_{p\pi\pi}$
				and $m_{p\pi}$ denote the reconstructed invariant masses
				of the proton-pion-pion and proton-pion systems, respectively
				and $m_{\Xi}$, $m_{\Lambda}$ are the PDG tabulated masses of 
				$\Xi$ and $\Lambda$, respectively.
		\end{itemize}
	\item {\bf Double tag}
		\begin{itemize}
			\item The $\pi^0$ candidate is reconstructed from a pair of photons
				which survive the good photon selections. An unconstrained 
				mass $M(\gamma\gamma)$ is calculated from energies and momenta
				of two photon pairs and it must be within $M(\pi^0) - 0.06 < 
				M(\gamma\gamma) < M(\pi^0) + 0.04 $
				. A kinematic fit of the two photons
				is also performed with a constraint $M(\gamma\gamma) = M(\pi^0)$.
				The $chi^2$ from kinematic fit must be less than 25.
				And the resulting energies and momenta of $\pi^0$ is saved for 
				further analyses.
			\item A kinematic fit with the following constraints is performed to suppress
				background and improve the resolution especially for final states in
				neutral leg,

				\begin{subequations}
					\begin{align}
						P_{J/\psi} = P_{\Xi^{\pm}} + P_{\pi^{\mp} + P_{\gamma 1}}
						+ P_{\gamma 2} + P_{n/\bar{n}}, \\
						M(\pi^0) = M(\gamma_1 \gamma_2),\\ 
						M(\Lambda/\bar{\Lambda}) = M(\gamma_1 \gamma_2 n/\bar{n}),
					\end{align}
				\end{subequations}
				where $P$ stands for the four-momenta, $M$ stands for the invariant mass,
				$\gamma_1$ and $\gamma_2$ is the photon from $\pi^0$ in energy descending order.
				The $\chi^2$ from the kinematic fit is required to be less than 200.
		\end{itemize}
	\item {\bf An additional requirement for neutron channel}
		\begin{itemize}
			\item As mentioned before, the shape of neutron shower is very similar with 
				one of a photon shower. The neutron shower has a possibility to pass 
				the good photon selections and form a $\pi^0$ candidate with a small
				energy photon. As shown in Fig.~\ref{fig:angleLNT}, the true opening 
				angle between $\Lambda$ and $n$ (or, $\bar{\Lambda}$ and $\bar{n}$)
				is less then 0.25~rad. Figure~\ref{fig:angleLN} show the open angle
				between $\Lambda$ and photons after reconstruction. For the neutron 
				channel, the peaks at 0.15~rad is caused by neutron shower being mistaken
				for photon shower. There is no such problem for anti-neutron 
				channel. In order to void the mistaken of neutron showers, an angle 
				cut for $\Lambda$ and the EMC showers is required. 
				When we select photons from the EMC shower queues,
				with four momenta of $\Lambda$ which is obtained by the energy-momentum 
				conservation in the recoiling system of $\bar{\Xi}$ and $\pi^-$, the 
				open angle between $\Lambda$ and the EMC showers can be calculated and
				will be required to be less than $15^\circ$.
		\end{itemize}

	\begin{figure*}[hp]
		\centering
		\mbox
		{
			\begin{overpic}[width=0.8\textwidth]{Figure/angleLNT.eps}
			\end{overpic}
		}
		\caption{}
		\label{fig:angleLNT}
	\end{figure*}

	\begin{figure*}[hp]
		\centering
		\mbox
		{
			\begin{overpic}[width=0.8\textwidth]{Figure/angleLN.eps}
			\end{overpic}
		}
		\caption{}
		\label{fig:angleLN}
	\end{figure*}





\end{itemize}

\subsection{Further selection}
After the initial selection criteria one is left with a sample that needs to 
be polished further. In order to reduce background contributions and reduce
data-Monte Carlo discrepancies, the following selection criteria have been 
applied. 
\begin{itemize}
	\item Further selection criteria
		\begin{itemize}
			\item requiring that the reconstructed decay lengths
				of all final state hyperons are greater than
				0;
			\item requiring that the cosine of the angle of the 
				reconstructed $\Xi^-$ in the center-of-mass
				frame, $\cos\theta_{\Xi^-, {\rm CM}}$, 
				fulfills the requirement $|\cos\theta_{\Xi^-, {\rm CM}}| < 0.84$.
			\item setting a mass window selection criteria for the
				$\Lambda$ candidates. For the neutron channel,
				we require that $|M(\bar{p}\pi^+)| < 0.0115~\GeV/c^2$, 
				for the anti-neutron channel,
				$|M(p\pi^-)| < 0.0115~\GeV/c^2$.
			\item setting a mass window selection criteria for the 
				$\Xi^-$ and $\bar{\Xi}^+$ candidates. For the 
				neutron channel, we require that
				$|M(n\gamma\gamma\pi^-)| < 0.011~\GeV/c^2$,
				$|M(\bar{p}\pi^+\pi^+)| < 0.011~\GeV/c^2$;
				for the anti-neutron channel,
				$|M(\bar{n}\gamma\gamma\pi^+)| < 0.011~\GeV/c^2$,
				$|M(p\pi^-\pi^-)| < 0.011~\GeV/c^2$.
		\end{itemize}
\end{itemize}






