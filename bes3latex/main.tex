%%%%%%%%%%%%%%%%%%%%%%%%%%%%%%%%%%%%%%%%%%%%%%%%%%%%%%%%%%%%%%%%%%%%%%%%%%%%%%
\documentclass[11pt, a4paper]{besnote}
\usepackage{besphysics}
\usepackage{besrefblk}
\usepackage{authblk}
\usepackage{color}
\usepackage{amsmath}
\usepackage{epsfig}
\usepackage{multirow}
\usepackage{overpic}
\usepackage{colortbl}

\usepackage{besphysics}
\usepackage{besrefblk}
\usepackage{authblk}
\usepackage{color}
\usepackage{amsmath}
\usepackage{comment}
\usepackage{mathrsfs}

\usepackage{amssymb}
\usepackage{threeparttable}
\usepackage{listings}
\usepackage{epsfig}
\usepackage{epstopdf}
\usepackage{longtable}
\usepackage{multirow}
\usepackage{overpic}
\usepackage{colortbl}
\usepackage{blindtext}
\usepackage{enumitem}
\usepackage{mathrsfs}
\usepackage{scrextend}
\usepackage{subcaption}
\addtokomafont{labelinglabel}{\bfseries}

\usepackage{graphicx}
%\usepackage{hepunits}
\usepackage[english]{babel}
\usepackage{subcaption}
\usepackage{xcolor}
\usepackage{blindtext}
\usepackage{enumitem}
\usepackage{listings}
\usepackage{siunitx}
%\usepackage[driverfallback=dvipdfm]{hyperref}
\usepackage{hyperref}
\DeclareSIUnit\GeVovercsq{\GeV/c^2}
\usepackage{tikz-feynman} 
\usepackage{mathrsfs}
\usepackage{feynmp}
\DeclareGraphicsRule{*}{mps}{*}{}
\usepackage{graphicx} % Allows including images
\usepackage{booktabs} % Allows the use of \toprule, \midrule and \bottomrule in tables
\usepackage{scrextend}
\usepackage{siunitx}
\usepackage{overpic}
\usepackage{amsmath}
\usepackage{eso-pic,graphicx}
\usepackage{pgf}
\usepackage{ragged2e}
\usepackage{tikz}
\usepackage{pict2e}
\usepackage{tikz-3dplot}
\usepackage{xpicture}
\usepackage{xcolor}
\usepackage{xcolor, soul}
\usepackage[most]{tcolorbox}

%\usepackage[colorlinks,linkcolor=blue,anchorcolor=blue,citecolor=blue,driverfallback=dvipdfm]{hyperref}
%\usepackage{hyperref}
%\usepackage[colorlinks,linkcolor=blue,anchorcolor=blue,citecolor=blue,dvipdfm]{hyperref}

\newcommand\T{\rule{0pt}{2.6ex}}       % Top strut
\newcommand\B{\rule[-1.2ex]{0pt}{0pt}} % Bottom strut
\newcommand{\xim}{\Xi^{-}}
\newcommand{\xibarp}{\bar{\Xi}^{+}}
\newcommand{\lam}{\Lambda}
\newcommand{\lambar}{\bar{\Lambda}}
\newcommand{\g}{\gamma}
\newcommand{\CP}{CP}
%\rennewcommand{\thefootnote}{\arabic{footnote}}
\newcommand{\F}{h}
\newcommand{\thetap}{\theta_1}

 \uchyph=0
 \lefthyphenmin=2
 \righthyphenmin=2 

%%%%%%%%%%%%%%%%%%%%%%%%%%%%%%%%%%%%%%%%%%%%%%%%%%%%%%%%%%%%%%%%
% for the title
\title{Measurement of decay parameters $\alpha_0 /\bar{\alpha}_0$ of $\Lambda/\bar{\Lambda}$ in the
process $J/\psi \to \Xi^- \bar{\Xi}^+$}
% for the authors
\author[a,b]{~Liang Liu 1}
\author[a]{~author 2}
\author[b]{~author 3}
\affil[a]{\it Institute of High Energy Physics, CAS}
\affil[b]{\it affil 1}
\affil[c]{\it Department of Computer Science and Engineering}

% Add the referee committee here
\refmember[d]{~Ref1 xx (Chair)}
\refmember[e]{~Ref2 xx}
\refmember[f]{~Ref3 xx}
\refaffil[d]{\it Department of Computer Science and Engineering}
\refaffil[e]{\it Department of Electrical Engineering}
\refaffil[f]{\it Latex Univeristy}

% for the draft version
\besmemoversion{1.0}

% Put the analysis memo docdb ID here
% Find the ID on the docdb page of the memo
% For example: DocDB-497 in http://docbes3.ihep.ac.cn/cgi-bin/DocDB/ShowDocument?docid=497
\besdocdbid{497}

% The analysis hypernews ID information
% Find the ID in the corresponding hypernews forum 
% For example: BAM-228 in http://hnbes3.ihep.ac.cn/HyperNews/get/paper228.html
\besmemoid{228}

% add the abstract of the note here.
\abstracttext{
  Based on 567 $\ipb$ of $\ee$ annihilation data collected with the BESIII detector at the
BEPCII produced at $\sqrt{s}=4.599\gev$,

.

.

.

.

.

.

.

.

.

.

.

.

.

.

.

.
}
%%%%%%%%%%%%%%%%%%%%%%%%%%%%%%%%%%%%%%%%%%%%%%%%%%%%%%%%%%%%%%%%%%%%%%%%%%%%%%
\begin{document}
\setlength{\baselineskip}{0.7cm}
%===========================================================================
%================== Table of contents ======================================
%===========================================================================
\tableofcontents

%===========================================================================
%================== context and BESIII memo ================================
\section{Introduction}

From the fundamental point of view, all matter is built out of fermions (spin-1/2 particles) 
with quarks being the elementary constituents. The antimatter in the modern theory was first 
predicted by Paul Dirac and discovered by Carl D. Anderson. Cosmological observations tell us
that our universe contains more matter than antimatter. The asymmetry between matter and 
antimatter can be characterized in terms of the baryon-to-photon ratio
\begin{equation}
	\eta \equiv \frac{n_{B} - n_{\bar{B}}}{n_{\gamma}} \approx 6\times10^{-10}.
\end{equation}
The physical process responsible for the asymmetry is called baryogengesis. To discover
the mechanism behind baryogengesis is one of the most important unresolved problems in 
fundamental physics. For now, it is well know that there necessary conditions for 
baryogengesis which is called Sakharov's conditions.
\begin{enumerate}
	\item B violation
	\item Loss of thermal equilibrium
	\item C, CP violation
\end{enumerate}

In 1956, the idea of testing the violation of parity (P) symmetry was proposed by 
Tsung-Dao Lee and Chen-Ning Yang firstly. The product of two transformations
charge conjugation (C) and parity (P) is the true symmetry between matter and 
antimatter. The Kobayashi-Maskawa mechanisam is the only confirmed way of CP violation predicted
by the Standard Model. The measuremet of the meson decays show that the Kobayashi-Maskawa
is, very likely, the dominant source of CP violation in low-energy flavor-changing 
processes. However, it predicts present baryon number density that is many orders of
magnitude below the cosmological observations, which indicates that there must exist
sources of CP violation beyong the Kobayashi-Maskawa phase in our Universe.

The hadronic decay of hyperons proceeds into both parity-violating ($S$-wave) and 
parity-conserving ($P$-wave) final states with amplitudes $S$ and $P$. The amplitude can
be written as
\begin{equation}
	{\rm Amp}(B\to b\pi) = S + P {\bf \sigma} \cdot {\bf q}
\end{equation} 
The experimental observables are the total decay width $\Gamma$, and the normalized decay asymmetry 
parameters $\alpha$, $\beta$, and $\gamma$.
\begin{equation}
	\begin{aligned}
		&\alpha^2 + \beta^2 +\gamma^2 = 1, \\
		&\alpha = 2 {\rm Re} (S^*P) / (|S|^2 + |P|^2), \\
		&\beta = 2 {\rm Im} (S^*P) / (|S|^2 + |P|^2). \\
	\end{aligned}
\end{equation}
Only two of these three parameters are independent.
 So, the decay parameters are usually parametrized by using two 
 essentially independent parameters $\alpha$ and $\phi$
\begin{equation}
	\begin{aligned}
		\beta &= \sqrt{1-\alpha^2} \sin \phi,\\
		\gamma &= \sqrt{1-\alpha^2} \cos \phi,
	\end{aligned}
\end{equation}
which are more closely related to experimental measurement. Using $\alpha$
and $\phi$, two CP violation observables 
$A_{CP} = \frac{\alpha + \bar{\alpha}}{\alpha - \bar{\alpha}}$
and 
$\phi_{CP} = \frac{\phi + \bar{\phi}}{2}$ can be defined. 
If CP conservation holds, $A_{CP}$ and $\phi_{CP}$ will be strictly equal to 0.
Any nonzero value of $A_{CP}$ and $\phi_{CP}$ indicates the CP violation in
hyperon decay.

In this work, we will focus on the following two decay channel,
\begin{subequations}
	\label{eq:xixi}
	\begin{align}
		e^+e^- &\to J/\psi \to \Xi^-\bar{\Xi}^+ \to \Lambda(\to n\pi^0)\pi^- \bar{\Lambda} (\to \bar{p}^-\pi^+)\pi^+, (\text{neutron channel}) \label{eq:neutronchannel}\\
		e^+e^- &\to J/\psi \to \Xi^-\bar{\Xi}^+ \to \Lambda(\to p^+\pi^-)\pi^- \bar{\Lambda} (\to \bar{n}\pi^0)\pi^+. (\text{anti-neutron channel}) \label{eq:antineutronchannel}
	\end{align}
\end{subequations}
Figure~\ref{fig:xixidig} shows the diagram of the decay channels.
The hadronic decays of two hyperons $\Xi^-$ (S = 2) and $\Lambda$ (S = 1) will be
studied. There are total 10 parameters in this process, two production parameter
$\alpha_{J/\psi}$ and $\Delta \Phi$, four decay asymmetry parameters of $\Xi^-$,
$\alpha^{\Xi}$, $\phi^{\Xi}$, $\bar{\alpha}^{\Xi}$, and $\bar{\phi}^{\Xi}$,
four decay asymmetry parameters of $\Lambda$, $\alpha^{\Lambda}_{-}$, 
$\bar{\alpha}^{\Lambda}_{+}$, $\alpha^{\Lambda}_{0}$, and $\bar{\alpha}^{\Lambda}_{0}$.
The unique advantage of this work is that we can measure the four decay modes of 
$\Lambda$ decay simultaneously,
\begin{equation}
	\begin{aligned}
		\Lambda &\to p\pi^-,\\
		\Lambda &\to n\pi^0,\\
		\bar{\Lambda} &\to \bar{p}\pi^+,\\
		\bar{\Lambda} &\to \bar{n}\pi^0,
	\end{aligned}
\end{equation}
so that, the isospin amplitude can be studied by combining the four decay modes. 
For $\Lambda \to p \pi^-$, the S-wave and P-wave can be expressed as
\begin{equation}
	\begin{aligned}
		S(\Lambda_-) = -\sqrt{\frac{2}{3}} S_{11} e^{i(\delta_{11}^S + \phi_1^S)} 
		+ \sqrt{\frac{1}{3}} S_{33} e^{i(\delta_{33}^S + \phi_3^S)},\\
		P(\Lambda_-) = -\sqrt{\frac{2}{3}} P_{11} e^{i(\delta_{11}^P + \phi_1^P)} 
		+ \sqrt{\frac{1}{3}} P_{33} e^{i(\delta_{33}^P + \phi_3^P)},
	\end{aligned}
\end{equation}
for $\Lambda \to n \pi^0$,
\begin{equation}
	\begin{aligned}
		S(\Lambda_0) = \sqrt{\frac{1}{3}} S_{11} e^{i(\delta_{11}^S + \phi_1^S)} 
		+ \sqrt{\frac{2}{3}} S_{33} e^{i(\delta_{33}^S + \phi_3^S)},\\
		P(\Lambda_0) = \sqrt{\frac{1}{3}} P_{11} e^{i(\delta_{11}^P + \phi_1^P)} 
		+ \sqrt{\frac{2}{3}} P_{33} e^{i(\delta_{33}^P + \phi_3^P)},
	\end{aligned}
\end{equation}
where $S$ and $P$ is isospin amplitudes with subscript convention 
$S_{2\Delta I, 2I}$ and $P_{2\Delta I, 2I}$. The average of $\alpha^{\Lambda}$ 
for two modes is the same as the values in the $|\Delta I| = 1/2$ limit,
\begin{equation}
	\alpha^{\Lambda} := \frac{2\alpha^{\Lambda}_{-} + \alpha^{\Lambda}_{0}}{3} = 2 S_{11} P_{11} \cos(\phi_{1}^P - \phi_1^S).
\end{equation}
The first order correction of $|\Delta I| = 3/2$ is given as:
\begin{equation}
	\begin{aligned}
		\frac{\alpha_{-}^{\Lambda}-\alpha_{0}^{\Lambda}}{\alpha^{\Lambda}}&=\frac{3}{\sqrt{2}} \frac{\Delta \alpha_{3 / 2}}{\cos \left(\phi_{1}^{P}-\phi_{1}^{S}\right)}+3\left(2 s^{2}-1\right) \Delta_{\Lambda}, \\
\Delta \alpha_{3 / 2} &=p_{3}\left[\left(1-s^{2}\right) \cos \left(2 \phi_{1}^{P}-\phi_{1}^{S}-\phi_{3}^{P}\right)-s^{2} \cos \left(\phi_{1}^{S}-\phi_{3}^{P}\right)\right] \\ 
		&+s_{3}\left[s^{2} \cos \left(2 \phi_{1}^{S}-\phi_{1}^{P}-\phi_{3}^{S}\right)-\left(1-s^{2}\right) \cos \left(\phi_{1}^{P}-\phi_{3}^{S}\right)\right] 
	\end{aligned}
\end{equation}
where $s:= S_{11}$, $s_3 := S_{33}/S_{11}$, and $p_3 := P_{33}/P_{11}$. 
We can construct isospin averages of the observables from two isospin modes
to recover the results in the $|\Delta I| = 1/2$ limit and require a better 
precision,
\begin{equation}
	A^{\Lambda}_{CP} := \frac{2 A^-_{CP} + A^0_{CP}}{3} = - \tan (\delta^P_{11} - \delta^S_{11}) \tan (\phi^P_1 - \phi^S_1).
\end{equation}
The CP observables $A^{\Xi}_{CP}$ and $\phi_{CP}^{\Xi}$ of $\Xi^-$ can also be measured to have a cross check with
the charged channel measurement.

\begin{figure}
	\unitlength = 1mm
	\centering
	\begin{fmffile}{xixi}
		\begin{fmfgraph*}(120,75)
			\fmfleft{i1,i2}
			\fmfright{o1,o2,o3,o4,o5,o6}
			\fmflabel{$e^-$}{i1}
			\fmflabel{$e^+$}{i2}
			\fmflabel{$n(p^+)$}{o6}
			\fmflabel{$\bar{p}^-(\bar{n})$}{o1}
			\fmflabel{$\pi^0(\pi^-)$}{o5}
			\fmflabel{$\pi^-$}{o4}
			\fmflabel{$\pi^+$}{o3}
			\fmflabel{$\pi^+(\pi^0)$}{o2}
			%\fmflabel{$i\sqrt{\alpha}$}{v1}
			%\fmflabel{$i\sqrt{\alpha}$}{v2}
			\fmfblob{.08w}{v2}
			\fmf{fermion}{i1,v1,i2}
			\fmf{double_arrow}{o1,v5}
			\fmf{double_arrow}{v6,o6}
			\fmf{vanilla}{v5,o2}
			\fmf{vanilla}{v3,o3}
			\fmf{vanilla}{v4,o4}
			\fmf{vanilla}{v6,o5}
			\fmf{double_arrow, label=$\Lambda$}{v4,v6}
			\fmf{double_arrow, label=$\Xi^-$}{v2,v4}
			\fmf{double_arrow, label=$\bar{\Xi}^+$}{v3,v2}
			\fmf{double_arrow, label=$\bar{\Lambda}$}{v5,v3}
			\fmf{photon,label=$\gamma$}{v1,v7}
			\fmf{dbl_dashes,label=$J/\psi$}{v7,v2}
		\end{fmfgraph*}
	\end{fmffile}
	\caption{Using \texttt{feynmp}}
	\label{fig:xixidig}
\end{figure}


\subsection{Previous results}

The decay process $J/\psi \to \Xi^-\bar{\Xi}^+$ belongs to family $J/\psi\to Y\bar{Y}$, where $Y$ 
stands for hyperon $\Lambda$, $\Sigma$, and $\Xi$. These decay processes are published
 or ongoing at BESIII collaboration. The results of the charged channel of $J/\psi \to \Xi^-\bar{\Xi}^+$
 and $J/\psi \to \Lambda \bar{\Lambda}$ are list in Table~\ref{table:main}.

\begin{table}[hbtp]
\centering
\normalsize
\caption[]{{\bf Summary of results.}}


  \vspace{0.2cm}
  \renewcommand{\arraystretch}{1.3}
\begin{tabular}{lll}
  \hline  \hline
   Parameter  & \multicolumn{1}{c}{BESIII result} & \multicolumn{1}{c}{Previous result} \\
  \hline
  $\alpha_{\psi}$           & $\phantom{-}0.586\pm 0.012 \pm0.010$ &$\phantom{-}0.58\pm0.04 \pm 0.08$ \hfill \cite{Ablikim:2016iym}\\
  $\Delta\Phi$              & $\phantom{-}1.213\pm 0.046 \pm 0.016$~rad &\multicolumn{1}{c}{--} \\

  $\alpha_{\Xi}$           & $-0.376\pm0.007\pm0.003$&$-0.401\pm0.010$\hfill \cite{Zyla:2020zbs}\\
    $\phi_{\Xi}$             & $\phantom{-}0.011\pm0.019\pm0.009$ rad &$-0.037\pm0.014$~rad\hfill \cite{Zyla:2020zbs}\\
  $\overline{\alpha}_{\Xi}$        & $\phantom{-}0.371\pm0.007\pm0.002$ &\multicolumn{1}{c}{--}\\

  $\overline{\phi}_{\Xi}$          & $-0.021\pm0.019\pm0.007$ rad &\multicolumn{1}{c}{--}\\
  $\alpha^{\Lambda}_-$           & $\phantom{-}0.757 \pm 0.011 \pm 0.008$ &$\phantom{-}0.750 \pm 0.009\pm 0.004$\hfill \cite{Ablikim:2018zay}\\
  $\overline{\alpha}^{\Lambda}_+$          & $-0.763 \pm 0.011 \pm 0.007$ &$-0.758\pm 0.010 \pm 0.007$\hfill\cite{Ablikim:2018zay}\\
  $\alpha^{\Lambda}_0$           & \multicolumn{1}{c}{--} & $\phantom{-}0.74\pm0.05$\\
  $\overline{\alpha}^{\Lambda}_0$          & $-0.692 \pm 0.016 \pm 0.006$ & \multicolumn{1}{c}{--}\\
    \hline  \hline
    $\xi_P - \xi_S$      & $\phantom{-}(1.2\pm3.4\pm0.8)\times10^{-2}$~rad & \multicolumn{1}{c}{--} \\
    $\delta_P - \delta_S$  & $(-4.0\pm3.3\pm1.7)\times10^{-2}$~rad & $\phantom{-}(10.2\pm3.9)\times10^{-2}$ ~rad\hfill\cite{Huang:2004jp} \\
  \hline  \hline
  $A_{\rm CP}^{\Xi}$ & $\phantom{-}(6.0\pm13.4\pm5.6)\times10^{-3}$ & \multicolumn{1}{c}{--} \\
  $\Delta\phi_{\rm CP}^{\Xi}$ & $(-4.8\pm13.7\pm2.9)\times10^{-3}$ ~rad & \multicolumn{1}{c}{--} \\
  $A_{\rm CP}^{\Lambda}$ & $(-3.7\pm11.7\pm9.0)\times10^{-3}$  &$(-6\pm12\pm7)\times10^{-3}$ \hfill\cite{Ablikim:2018zay} \\
  \hline  \hline
  $\left<\phi_{\Xi}\right>$            & $\phantom{-}0.016 \pm 0.014 \pm 0.007$~rad & \\
  \hline  \hline
\end{tabular}
  \vspace{0.3cm}
 \label{table:main}
\end{table}



%\section{Formalism}
\label{sec:formalism}
Consider the production of the spin-$1/2$ hyperon-antihyperon pair $e^+e^- \rightarrow Y\overline{Y}$. The formalism presented here is derived assuming one-photon exchange and follows the modular approach, described in detail in Ref.~\cite{Perotti:2018wxm}. The spin projections of the produced $Y\bar{Y}$ are entangled and for spin-1/2 baryons the system is described by only two complex parameters - the hadronic form factors $G_E^{\psi}$ and $G_M^{\psi}$~\cite{GF16+}, just as in the case of electromagnetic production. In fact, since the global complex phase can not be observed, instead of two complex form factors 
three real parameters are commonly used: an overall factor corresponding to the modulus of the
hadronic magnetic form factor, the modulus of the form factor ratio $\rho$, and their relative phase 
$\Delta\Phi=\Phi_E^\Psi-\Phi_M^\Psi$. A non-zero phase between the form factors means that the produced particles are polarized, even if the colliding beams are unpolarized~\cite{AZ96}. The polarization is perpendicular to the reaction plane and given by a simple function of the scattering angle. It can be determined from the asymmetries of the hyperon two-body decays.

\subsection{Asymmetry decay parameters}
For a hyperon of spin 1/2 which subsequently decays weakly to a baryon of spin 1/2 and a pseudoscalar spin-0 meson, e.g. $\lam \rightarrow p \pi^{-}$, the conservation of total angular momentum requires that the final states are either in relative angular momentum 0 or 1, denoted $s$ or $p$ wave, respectively. The decay distributions are described by the asymmetry parameter $\alpha$ of the baryon angular distribution, and the baryon transverse polarization using parameters $\beta$ and $\gamma$~\cite{MS08}. The decay parameters are unique to each particular decay mode. Experimentally $\alpha$ is measured by the differential counting rate $dN/d\Omega \propto(1+\alpha \vec{P_{Y}} \cdot \vec{p}_{B})$, where $\vec{P_Y}$ is the hyperon polarization and $\vec{p}_{B}$ is the decay baryon momentum unit vector in the hyperon rest frame. Thus the $\alpha$-parameter describes the asymmetry in the angular distribution of the emitted baryon in the hyperon rest frame. The angular asymmetry also gives information about the polarization of the parent hyperon. In terms of the $s$ and $p$ amplitudes, 
\begin{equation}
    \alpha = 2Re(A_s^*\cdot A_{p} ), \ \ \beta=2Im(A_s^*\cdot A_{p}) \ \ \gamma=|A_{s}|^{2}-|A_p |^2.
\end{equation}
Experimentally $\beta$ and $\gamma$ are reconstructed from the measured observable $\phi_{Y}$,
\begin{equation}
    \label{betagamma}
    \beta=\sqrt{1-\alpha^{2}}\sin\phi_{Y}, \ \gamma = \sqrt{1-\alpha^2}\cos\phi_{Y}.
\end{equation}
If baryon-antibaryon symmetry holds, then the hyperon asymmetry parameter is equal and opposite to the one for the CP-odd anti-hyperon, $\alpha_{Y}=-\alpha_{\bar{Y}}$. The corresponding CP violating observable is defined as 
\begin{equation}
    \label{alpha}
    A_{\rm CP,Y} = \frac{\alpha_{Y} + \alpha_{\bar{Y}}}{\alpha_{Y}-\alpha_{\bar{Y}}}.
\end{equation}
$A_{\rm CP, Y}$ requires final state interactions in order not to vanish. A model-independent approximation of $A_{\rm CP, Y} = -\tan(\delta_p -\delta_s)\sin(\xi_p- \xi_s)$, where $(\delta_p -\delta_s)$ is the difference between the $p$- and $s$-wave final scattering phase shifts of the baryon-pseudoscalar decay products and $(\xi_p- \xi_s)$ is the CP-violating difference between weak interaction phases in the decay~\cite{MH04}. Limits on the observability of $A_{\rm CP, Y}$ can be determined from  $\tan^{-1}\phi_{Y}=\beta_{Y}/\gamma_{Y}=\tan(\delta_p -\delta_s)$. From our previous work we have good reasons to believe that final state interactions for the weak decay $\Xi\to\Lambda\pi$ is very small as the average value of $\phi_{\Xi}$, $\left<\phi\right>_{\Xi}=0.016\pm0.014\pm0.007$ is consistent with 0. With a larger data set one may determine if it is non-zero.
Small final state interactions instead enhance the CP observable constructed from the $\beta$-parameter,  
\begin{equation}
    \label{beta}
    B_{\rm CP} = \frac{\beta_{Y} + \beta_{\bar{Y}}}{\beta_{Y}-\beta_{\bar{Y}}}=\frac{\sin\phi_{Y}+\sin\phi_{\bar{Y}}}{\sin\phi_{Y}-\sin\phi_{\bar{Y}}},
\end{equation}
and can therefore be and order of magnitudes more sensitive compared to $A_{\rm CP}$~\cite{JD85, JD86, MS08}. This is because $B_{\rm CP}=-\sin(\xi_p- \xi_s)/\tan(\delta_p -\delta_s)$.
The test on $B^_{CP}$ was proposed more than 30 years ago~\cite{JD85, JD86}. Experimentally, $B_{\rm CP}$ is not an optimal observable as it is not independent of $A_{\rm CP}$ (it contains both $\alpha_{\Xi}$ and $\phi_{\Xi}$) In addition, the denominator is very small and hence such observable would provide a null result event if the numerator significantly differs from $0$. On the other hand, what is really tested is how $\phi_{\Xi}$ relates to $\bar{\phi}_{\Xi}$. Hence, $\Delta\phi = (\phi_{\Xi} + \bar{\phi}_{\Xi})/2$ and $\left< \phi_{Y} \right> =  (\phi_{Y} - \bar{\phi}_{Y})/2$, the CP-test using the decay parameter of $\phi$ is tested via

\begin{equation}
    \label{deltaphi}
    \phi_{\rm CP} = \Delta\phi.
\end{equation}

In this work, in addition to $A_{\rm CP}$, $\phi_{\rm CP}$ is measured for the first time. 
\subsection{Strong and weak phase differences}
CP violation can only be observed if there is interference between CP even and CP odd terms in the decay amplitude. Since the decay amplitude $\Xi^- \to \Lambda \pi^-$ consists of both P-wave (parity conserving) and an S-wave (parity violating) part, the leading order contribution to the CP asymmetry $A_{CP}^{\Xi}$ can be written
    \begin{equation}
      A_{CP}^{\Xi}\sim - \tan(\delta_P - \delta_S)\sin(\xi_P - \xi_S),
      \label{eq:acpphase}
  \end{equation}
where $\delta_P-\delta_S = \beta/\alpha$ denotes the strong phase difference of the final state interaction between $\Lambda$ and $\pi^-$ from the $\xim$ decay. CP violating effects would manifest itself in a non-zero weak phase difference $\xi_P - \xi_S$  \cite{JD86}, an observable that is complementary to the kaon decay parameter $\epsilon^{'}$ 
%\cite{Christenson:1964fg, AlaviHarati:1999xp, Fanti:1999nm} 
since the latter only involves an $S$-wave. A non-zero $A_{CP}^{\Xi}$ requires both non-zero strong and weak phase contributions. Therefore,  to pinpoint the weak phase difference, an independent measurement of the strong phase is required. However, since the latter is extracted from the $\phi_\Xi$ parameter, that has been found to be small ($-0.037\pm0.014$\cite{MH04,PDG18}), CP violating signals in $A_{CP}^\Xi$ are strongly suppressed and difficult to interpret in terms of the weak phase difference.

An independent and direct CP symmetry test in $\Xi^-\to\Lambda\pi^-$ is provided by determining the  $\Delta \phi_{CP}$ value. In the leading order, it is related to the weak phase difference in the following way:
\begin{equation}
 (\xi_P - \xi_S)_{\rm{LO}} = \frac{\beta + \bar{\beta}}{\alpha - \bar{\alpha}}\approx \frac{\sqrt{1-\alpha^2}}{\alpha}\Delta \phi_{CP}.  
 \label{eq:betaprime}
\end{equation} 
This means that the sensitivity to CP violation effects is enhanced by an order of magnitude with respect to that of the $A_{CP}^{\Xi}$ observable\cite{JD85, JD86}. However, to measure 
$\Delta \phi_{CP}$ using standard techniques requires that the polarization is either precisely known or identical for $\Xi^-$ and $\overline{\Xi}^+$ in separate experiments.  In principle, this can be handled if one has access to huge data samples \cite{MS08}. However, the precision will be limited by systematic effects due to asymmetries in the production of the baryon and antibaryon. Here, we present an alternative approach where the baryon and the antibaryon are entangled CP eigenstates and analyzed simultaneously.

\subsection{Overall reference system}
\noindent We are interested in the process (\ref{e+e-toxixibar}), 
\begin{equation}
    e^{+}e^{-}\to J/\psi\rightarrow \xim\xibarp\to\lam\pi^{-}\lambar\pi^{+}\to p\pi^{-}\pi^{-}\bar{p}\pi^{+}\pi^{+}.
\end{equation}
%\begin{align}
%	\mathbf{p}_{\xim}  &=  - \mathbf{p}_{\xibarp}= \mathbf{p},  \\
%	\mathbf{k}_{e^-}  &= - \mathbf{k}_{e^+} = \mathbf{k}. 
%\end{align}

%\noindent The scattering plane is spanned by the $e^+$ and the outgoing $\xim$ where the normal is defined by $\hat{n}=\hat{\mathbf{p}}\times\hat{\mathbf{k}}$. The $\xim$ scattering angle is given by

%\begin{equation}
%    	\cos\theta_{\xim}= \hat{\mathbf{p}}\cdot \hat{\mathbf{k}}.
%	\end{equation}

%\begin{figure}[h]
%\begin{center}
%\includegraphics[width=1.\linewidth]{fig/formalism/process.png}
%\caption{The coordinate system of the $e^+e^-\to J/\psi\to\xim\xibarp$ process.}
%\label{fig:coordinatesystem}
%\end{center}
%\end{figure}

\noindent To describe the full differential cross section of the exclusively measured process the nine angles given in vector $\xi$ are needed, $\xi= ( \theta_{\xim}, \theta_{\lam}, \varphi_{\lam}, \theta_{\bar{\lam}}, \varphi_{\bar{\lam}}, \theta_{p}, \varphi_{p}, \theta_{\bar{p}}, \varphi_{\bar{p}})$. Besides $\theta_{\xim}$, the other eight observables are the polar and azimuthal helicity angles of the $\Lambda$, $\bar{\lam}$, proton and anti-proton, respectively~\cite{MA10, BAM_licui}. With the helicity angles one can estimate the eight unknown parameters of the process, $\Omega = (\alpha_{\psi}, \ \Delta\Phi, \ \alpha_{\lam}, \ \alpha_{\lambar}, \ \alpha_{\xim}, \ \alpha_{\xibarp}, \ \phi_{\Xi}, \ \bar{\phi}_{\xibarp})$. 

%From a right-handed coordinate system with the basis vectors
%\begin{equation}
%	\hat{\textbf{e}}_{x} = \frac{1}{\sin{\theta}}(\hat{\textbf{p}} \times \hat{\textbf{k}})\times \hat{\textbf{p}} , \ \
%	\hat{\textbf{e}}_{y} = \frac{1}{\sin{\theta}}(\hat{\textbf{p}} \times \hat{\textbf{k}}), \ \
%	\hat{\textbf{e}}_{z} = \hat{\textbf{p}},
%\end{equation}
%the helicity angles are obtained. As seen from the basis vector definitions the helicity angles are defined with respect to the direction of the outgoing hyperon in the CM frame ($\hat{e}_{z}$). 

%\noindent The density matrix of the produced $\Xi\bar\Xi$ system is given by \cite{Perotti:2018wxm}:
%\begin{equation}
%    \rho_{\Xi\bar \Xi}\propto \sum_{\mu,\bar\nu=0}^3C_{\mu\bar\nu}\sigma_\mu^\Xi\otimes \sigma_{\bar\nu}^{\bar\Xi},
%\end{equation}
%where the production matrix is given by Eqn.~(23) in  \cite{Perotti:2018wxm}: $C_{\mu\bar\nu}\to C_{\mu\bar\nu}(\theta_\Xi;\alpha_{J/\psi},\Delta\Phi)$.
%Using replacement Eqn.~(42) and finally trace for the unmeasured polarization 
%of the protons one obtains the final differential distribution in the form: 
%\begin{equation}
%    \frac{d\Gamma}{d\xi}\propto \sum_{\mu,\bar\nu=0}^3\sum_{\mu'=0}^3\sum_{\bar\nu'=0}^3C_{\mu\bar\nu}\ 
%    a_{\mu\mu'}^\Xi\ a_{\mu'0}^\Lambda\ 
%    a_{\bar\nu\bar\nu'}^{\bar\Xi}\ a_{\bar\nu'0}^{\bar\Lambda}
%\end{equation}
%where the $a_{\mu\nu}$ matrices for $1/2\to 1/2+0$ decays are given by Eqn.~(50) in Ref.~\cite{Perotti:2018wxm}. More explicitly the matrices are the function of the following 
%helicity variables and decay parameters: $a_{\mu\mu'}^\Xi\to a_{\mu\mu'}^\Xi(\theta_\Lambda,\phi_\Lambda; \alpha_{\Xi},\phi_\Xi)$,
% $a_{\bar\nu\bar\nu'}^{\bar\Xi}\to a_{\bar\nu\bar\nu'}^{\bar\Xi}(\theta_{\bar\Lambda},\phi_{\bar\Lambda}; \alpha_{\bar\Xi},\phi_{\bar\Xi})$,
% $a_{\mu'0}^\Lambda\to a_{\mu'0}^\Lambda(\theta_p,\phi_p; \alpha_{\Lambda})$ and $a_{\bar\nu'0}^{\bar\Lambda}\to a_{\bar\nu'0}^{\bar\Lambda}(\theta_{\bar p},\phi_{\bar p}; \alpha_{\bar\Lambda})$.
 
\subsection{Phenomenology description of \texorpdfstring{$e^+e^-\to J/\psi,\psi(2S)\to \Xi\bar\Xi$}{eeXiXi}}\label{sub:C}

 Suppose one has produced a ``mother'' particle that
decays further. One wants to change from the production frame of this state to its rest frame. 
Given the state's three-momentum 
\begin{equation}
  {\bf p}_m = 
p_m \, (\cos\varphi_m\sin\theta_m,\sin\varphi_m\sin\theta_m,\cos\theta_m)
\end{equation}
and the $z$-axis in the production frame, one possibility would be to 
perform a single rotation that aligns ${\bf p}_m$ with the $z$-axis. 
Subsequently one then boosts to the rest frame of the mother particle. 
The single rotation would be around an axis perpendicular to 
${\bf p}_m$ and $\hat {\bf z}$. Yet when viewed as rotations around 
the coordinate axes this amounts to a succession of three rotations. 
Viewed as active rotations these are 
(a) a rotation around the $z$-axis by $-\varphi_m$; (b) a rotation around 
the $y$-axis by $-\theta_m$; (c) a rotation around the $z$-axis by $+\varphi_m$; see also Ref.~\cite{Jacob:1959at}. 
In principle, however, the first two rotations are sufficient to align
${\bf p}_m$ with the $z$-axis. In line with the present BESIII analyses 
we follow this two-rotation procedure
in the present work. The rotation matrix for ${\bf p}_m$ is given by
\begin{equation}
\left(
\begin{array}{rrr}
 \cos\theta_m \cos\varphi_m & \cos\theta_m \sin\varphi_m & -\sin\theta_m \\
 -\sin\varphi_m & \cos\varphi_m & 0 \\
 \cos\varphi_m \sin\theta_m & \sin\theta_m \sin\varphi_m & \cos\theta_m \\
\end{array}
\right).\label{eq:hrot}
\end{equation}
This rotation defines in a unique way the helicity reference frame 
for a daughter particle.
In an experimental analysis the  boosts and rotations in Eqn.~\eqref{eq:hrot}
are applied recursively to all decay products of
a decay chain, thus
defining a set of  helicity variables
to describe an event. 

The production process $e^+e^-\to B_1\bar B_2$, 
viewed in the CM frame, defines a 
scattering plane and therefore a coordinate system. The $z$-axis is 
chosen along the line of flight  of the incoming
positron, i.e.\ ${\bf\hat z}={\bf p}_{e^+} = (0,0,p_{\rm in})$, where $p_{\rm in}$ 
denotes the modulus of the momentum of electron and positron in the 
CM frame. The $y$-axis is chosen to be perpendicular to 
the scattering plane. One uses the direction of the baryon $B_1$ to 
define the $y$-axis: 
\begin{eqnarray}
  \label{eq:def-initial-y}
  {\bf\hat y} := \frac{{\bf p}_{e^+} \times {\bf p}_B}%
  {\vert {\bf p}_{e^+} \times {\bf p}_B \vert}  \, .
\end{eqnarray}
Finally the $x$-axis is chosen such that $x$, $y$ and $z$ adhere to the 
right-hand rule. Thus, the general coordinate system is given by

\begin{eqnarray}
  \label{eq:def-initial-xyz}
  {\bf\hat x} := {\bf\hat y}\times{\bf\hat z}, \
  {\bf\hat y} := \frac{{\bf p}_{e^+} \times {\bf p}_B}
  {\vert {\bf p}_{e^+} \times {\bf p}_B \vert}  , \ {\bf\hat z} := \frac{{\bf p}_{e^+}}{\vert {\bf p}_{e^+} \vert } .
\end{eqnarray}

Denoting the scattering angle of $B_1$ 
by $\theta_1$, all this implies that the baryon three-momentum is
${\bf p}_B = p_{\rm out} \, (\sin\theta_1,0,\cos\theta_1)$. 
Here $p_{\rm out}$ denotes the modulus of the 
momentum of baryon and antibaryon in the CM frame.
\begin{figure}
\centering
%\includegraphics[width=0.48\textwidth,bb=0 0 551 287]{axesBB.pdf}
\includegraphics[width=1.0\textwidth]{fig/formalism/axesBB.png}
\caption[]{(color online) Orientation of the axes in baryon $B_1$ and
  antibaryon $\bar B_2$ helicity frames.}
  \label{fig:axes}
\end{figure}
  
With the above definition of the CM coordinate system,
the $y$-axis of
the helicity frame of the baryon $B_1$, ${\bf\hat y}_1$ in Fig.~\ref{fig:axes}, is the same as ${\bf\hat y}$ in
Eqn.~\eqref{eq:def-initial-y}.
Therefore, for the helicity rotation matrix
Eqn.~\eqref{eq:hrot} one uses $\theta_m =\theta_1$ and $\varphi_m =0$.
Correspondingly, to transform to the helicity frame of the antibaryon $\bar B_2$
one chooses $\varphi_m=\pi$ and $\theta_m= \pi-\theta_1$. In this way 
the $y$-axis, ${\bf\hat y}_2$, is equal $-{\bf\hat y}$. The
$y$- and $z$-axes of the helicity frames of the baryon $B_1$ and the antibaryon
$\bar B_2$ have opposite directions while it is the same direction for the $x$-axis as shown in Fig.~\ref{fig:axes}. The coordinate system for the $B_1$ are given in the spherical coordinate system where the radial component is in the momentum direction of the $\xim$, $\theta$

Thus for the baryon $B_1$ helicity frame the axes for $B_1$ are defined as
\begin{eqnarray}
  \label{eq:def-B1-xyz}
  {\bf\hat x}_1 := {\bf\hat r}_1\cos\theta_{\Lambda}\sin\varphi_{\Lambda}, \
  {\bf\hat y}_1 := {\bf\hat r}_1\sin\theta_{\Lambda}\sin\varphi_{\Lambda}, \ 
  {\bf\hat z}_1 := {\bf\hat r}_1\cos\theta_{\Lambda} := \frac{{\bf p}_{\xim}}{\vert {\bf p}_{\xim} \vert }.
\end{eqnarray}


It is well known how the spin density matrices look like for a
reaction $e^+e^-\to B_1\bar{B}_2$ where both produced particles have
spin 1/2. The results were obtained using different approaches
\cite{AZ96,Czyz:2007wi,TomasiGustafsson:2005kc,Faldt:2013gka,Faldt:2016qee,Faldt:2017kgy} but here we use expressions from Ref.~\cite{Perotti:2018wxm} compatible with the introduced helicity variables.
The $\Xi^-$ has positive parity $\eta_1=1$ and the $\bar\Xi^+$
negative parity $\eta_2=-1$.  In general only two out of four possible
helicity transitions are independent. Using $\eta_1\eta_2 =-1$ for the
baryon antibaryon pair one can set $A_{1/2,1/2}=A_{-1/2,-1/2}=:\F_1$
and $A_{1/2,-1/2}=A_{-1/2,1/2}=:\F_2$.  The transition amplitude
matrix is
\begin{equation}
\left(
\begin{array}{cc}
 {\F_1} & {\F_2} \\
 {\F_2} & {\F_1} \\
\end{array}
\right)\, .
\end{equation}

The spin density matrix for a two-particle $1/2+\overline{1/2}$ system can be expressed
in terms of a set of $4\times 4 $ matrices obtained
from the outer product, $\otimes$, of $\sigma_\mu$ and ${\sigma}_{\bar\nu}$ \cite{Tabakin:1985yv}:
\begin{equation}
  \rho_{B_1,\bar B_2}=\frac{1}{4}\sum_{\mu,\bar\nu=0}^3C_{\mu\bar\nu}(\thetap\!)\,
  \sigma_\mu^{B_1}\otimes
      {\sigma}_{\bar\nu}^{\bar B_2},
\label{eqn:sig12}
\end{equation}
{where $\sigma_\mu^{B}$ with $\mu=0,1,2,3$ represent spin-$1/2$
  base  matrices
  for a  baryon $B$ in the rest frame. The $2\times 2$ matrices
 are $\sigma_0^{B}=\mathds{1}_2$, 
$\sigma_1^{B}=\sigma_x$, $\sigma_2^{B}=\sigma_y$ and $\sigma_3^{B}=\sigma_z$.
In particular the spin matrices $\sigma_\mu^{B_1}$ and
${\sigma}_{\bar\nu}^{\bar B_2}$ are given in the helicity frames of
the baryons $B_1$ and $\bar B_2$, respectively. The axes of the frames
are defined in Fig.~\ref{fig:axes} and denoted by
${\bf\hat x}_1,{\bf\hat y}_1,{\bf \hat z}_1$ and ${\bf\hat x}_2,{\bf\hat y}_2,{\bf\hat z}_2$.
  The real coefficients $C_{\mu\bar\nu}$
  are functions of the scattering angle $\thetap$ of $B_1$.
}

Suppose one is not interested in the absolute size of the cross section but only in the 
(not normalized) angular distributions. For their description we do not need all information
contained in the two complex form factors $\F_1$ and $\F_2$. Instead we can use just two real
parameters: first, $\alpha_\psi$ as defined below and, second, the relative phase between the form
factors $\Delta\Phi=\arg(\F_1/\F_2)$, i.e.\ we disregard the normalization and 
the overall phase.  More specifically without any loss of generality we take $\F_1$ as
real and set $\F_1=\sqrt{1-\alpha_\psi}/\sqrt{2}$ and
$\F_2=\sqrt{1+\alpha_\psi}\exp(-i\Delta\Phi)$.
Only 8 coefficients $C_{\mu\bar\nu}$ are non-zero and they are given by
\begin{eqnarray}
C_{00}&=&2(1+\alpha_\psi\cos^2\!\thetap\!)   \,, \nonumber\\
C_{0 2}&=&2\sqrt{1-\alpha_\psi^2}\sin\thetap\cos\thetap\sin(\Delta\Phi)   \,, \nonumber\\
C_{1 1}&=&2\sin^2\!\thetap   \,, \nonumber\\
C_{1 3}&=&2\sqrt{1-\alpha_\psi^2}\sin\thetap\cos\thetap\cos(\Delta\Phi)   \,, \nonumber\\
C_{20}&=&-C_{0 2}   \,,\label{eqn:c1212} \\
C_{2 2}&=&\alpha_\psi C_{11}   \,, \nonumber\\
C_{3 1}&=&-C_{1 3}   \,, \nonumber\\
C_{3 3}&=&-2(\alpha_\psi+\cos^2\!\thetap\!)\, .\nonumber
\end{eqnarray}

 


 
\subsection{Decay chains}
\label{sec:decaychains}


Since the joined production density matrix of Eqn.~\eqref{eqn:sig12}
is expressed as outer products of the basis matrices $\sigma_\mu$ it
is enough to know how the latter individually transform under a decay
process. This allows for modular expressions for the 
angular distributions and we follow Ref.~\cite{Perotti:2018wxm}.

Let us consider the weak decay mode of a spin-$1/2^+$ hyperon decaying
into a spin-$1/2^+$
hyperon and a pseudoscalar meson. The angular distribution is specified by two angles $\theta$ and $\phi$, which give the direction of the
final baryon in the helicity frame of the initial hyperon.  The spin
configuration of the final system is fully specified by the spin density
matrix of the final baryon, which has spin $1/2$, since
the accompanying particle is a pseudoscalar meson.
We introduce 
a $4\times 4$ matrix $a_{\mu\nu}$ which allows to express the
$\sigma_\mu$ matrices in the mother helicity frame in terms of $\sigma_\nu^d$
matrices in the daughter helicity frame:  
\begin{equation}
\sigma_\mu\to\sum_{\nu=0}^3 a_{\mu\nu}\sigma_\nu^d\, . \label{eqn:decay12p}
\end{equation}
The decay matrix $a_{\mu\nu}$ introduced above 
allows to keep track of the spin correlation between the decay 
products of the $B_1$ and $\bar B_2$ decays chains.

Below we provide the explicit expression for the decay matrices
$a_{\mu\nu}$. The transition amplitude is:
\begin{equation}
{}_{\rm d}\langle \theta, \varphi, \lambda\vert S \vert 0,0,\kappa\rangle
_{\rm m}
\propto {\cal D}^{J *}_{\kappa,\lambda}(\Omega) B_{\lambda},
\label{eqn:amphdecay}
\end{equation}
where ${\cal D}^{J }_{\kappa,\lambda}(\Omega)={\cal D}^{J }_{\kappa,\lambda}(\varphi,\theta,0)$ is Wigner D function (see remark in concerning Wigner D function definition in Ref.~\cite{Perotti:2018wxm}) and $\theta$, $\phi$
are spherical coordinates of the daugther baryon in the mother baryon helicity frame.
The coefficients  $a_{\mu\nu}$ are then obtained by multiplying the amplitude above
by its conjugate and inserting basis $\sigma$ matrices for the mother 
and the daughter baryon.
These coefficients can be rewritten  in terms of the 
decay parameters $\alpha_D$ and $\phi_D$ defined in Ref.~\cite{PDG18}. For completeness 
we first relate the helicity amplitudes to the $s$ and $p$
wave amplitudes $A_s$ and $A_p$, corresponding respectively to the 
parity violating and parity conserving transitions. 
If a hyperon of spin $J$ decays (weakly) into a hyperon of spin $S$ and a (pseudo)scalar state, then the relation between 
helicity amplitudes and canonical amplitudes is given by \cite{Jacob:1959at}
\begin{eqnarray}
  \label{eq:hel-LS}
  B_\lambda = \sum\limits_L \left(\frac{2L+1}{2J+1}\right)^{1/2} \, (L,0;S,\lambda\vert J,\lambda) \, A_L 
\end{eqnarray}
where $(s_1,m_1,s_2,m_2\vert s, m)$ is a Clebsch-Gordan coefficient.
For $J=S=1/2$ the helicity amplitudes are\footnote{Note that the Particle Data Group \cite{PDG18} uses $-A_p=A_p^{\rm PDG}$.} 
\begin{eqnarray}
B_{-1/2}&=&\frac{A_s+A_p}{\sqrt{2}} \,, \nonumber \\
B_{1/2}&=&\frac{A_s-A_p}{\sqrt{2}} \,.
\label{eqn:JWdec}
\end{eqnarray} 
Using the normalization $|A_s|^2+|A_p|^2=|B_{-1/2}|^2+|B_{1/2}|^2=1$,
the relation between helicity amplitudes and the decay parameters
is:
\begin{eqnarray}
\alpha_D&=&-2\Re(A_s^*A_p)=|B_{1/2}|^2-|B_{-1/2}|^2  \,, \nonumber\\
\beta_D&=&-2\Im(A_s^*A_p)=2\Im(B_{1/2}B_{-1/2}^*)  \,,\label{eqn:dparam} \\
\gamma_D&=&|A_s|^2-|A_p|^2=2\Re(B_{1/2}B_{-1/2}^*),\nonumber   
\end{eqnarray}
where $\beta_D=\sqrt{1-\alpha_D^2}\sin\phi_D$ and 
 $\gamma_D=\sqrt{1-\alpha_D^2}\cos\phi_D$.
The non-zero elements  of the decay matrix $a_{\mu\nu}(\theta,\varphi;\alpha_D,\beta_D,\gamma_D)$  are: 
\begin{eqnarray}
a_{00}&=&1 \,, \nonumber\\
a_{03}&=&\alpha_D \,, \nonumber\\
a_{10}&=&\alpha_D\cos\varphi\sin\theta \,, \nonumber\\
a_{11}&=&\gamma_D \cos\theta\cos\varphi-\beta_D \sin\varphi \,, \nonumber\\
a_{12}&=&-\beta_D \cos\theta
\cos\varphi-\gamma_D \sin\varphi \,, \nonumber\\
a_{13}&=&\sin\theta \cos\varphi \,, \nonumber\\
a_{20}&=&\alpha_D\sin\theta \sin\varphi \,, \label{eqn:matrixa}\\
a_{21}&=&\beta_D \cos\varphi+\gamma_D \cos\theta \sin\varphi \,, \nonumber\\
a_{22}&=&\gamma_D\cos\varphi-\beta_D \cos\theta \sin\varphi \,, \nonumber\\
a_{23}&=&\sin\theta \sin\varphi \,, \nonumber\\
a_{30}&=&\alpha_D\cos\theta \,, \nonumber\\
a_{31}&=&-\gamma_D\sin\theta \,, \nonumber\\
a_{32}&=&\beta_D\sin\theta \,, \nonumber\\
a_{33}&=&\cos\theta\, .\nonumber
\end{eqnarray}


Here we discuss an exclusive decay chain: $e^+e^-\to J/\psi,\psi(2S)\to\Xi\bar\Xi $ where $\Xi(\bar\Xi)$
decays weakly:
$\Xi(\bar\Xi)\to\Lambda(\bar\Lambda)\pi$ and then $\Lambda(\bar\Lambda)$ decays weakly: $\Lambda\to p\pi^-(\bar\Lambda\to \bar p\pi^+)$.
The production spin density matrix is given by Eqn.~\eqref{eqn:c1212}: $C_{\mu\bar\nu}\to C_{\mu\bar\nu}(\theta_\Xi;\alpha_{\psi},\Delta\Phi)$.
Using replacements Eqn.~\eqref{eqn:decay12p} for the sequential decays
and finally taking trace for the unmeasured polarization 
of the final proton-antiproton system
one obtains the differential distribution in the form: 
\begin{equation}
\mathrm{Tr}\rho_{p\bar{p}}\propto {\cal W}=\sum_{\mu,\bar\nu=0}^3C_{\mu\bar\nu}\sum_{\mu'=0}^3\sum_{\bar\nu'=0}^3
    a_{\mu\mu'}^\Xi\ a_{\mu'0}^\Lambda\ 
    a_{\bar\nu\bar\nu'}^{\bar\Xi}\ a_{\bar\nu'0}^{\bar\Lambda}\ ,\label{eq:XiXi1}
\end{equation}
where the $a_{\mu\nu}$ matrices for $1/2\to 1/2+0$ decays are given by Eqn.~\eqref{eqn:matrixa}. The matrices are the functions of the corresponding 
helicity variables and decay parameters: $a_{\mu\mu'}^\Xi\to a_{\mu\mu'}^\Xi(\theta_\Lambda,\varphi_\Lambda; \alpha_{\Xi},\beta_\Xi,\gamma_\Xi)$,
 $a_{\bar\nu\bar\nu'}^{\bar\Xi}\to a_{\bar\nu\bar\nu'}^{\bar\Xi}(\theta_{\bar\Lambda},\varphi_{\bar\Lambda}; \alpha_{\bar\Xi},\beta_{\bar\Xi},\gamma_{\bar\Xi})$,
$a_{\mu'0}^\Lambda\to a_{\mu'0}^\Lambda(\theta_p,\varphi_p; \alpha_{\Lambda})$ and $a_{\bar\nu'0}^{\bar\Lambda}\to a_{\bar\nu'0}^{\bar\Lambda}(\theta_{\bar p},\varphi_{\bar p}; \alpha_{\bar\Lambda})$.
Therefore Eqn.~\eqref{eq:XiXi1} could be rewritten as
\begin{equation}
  {\cal W}=\sum_{\mu,\bar\nu=0}^3C_{\mu\bar\nu}(\theta_\Xi;\alpha_{\psi},\Delta\Phi)Y_{\mu\bar\nu}(\boldsymbol{\xi};\Omega),  
\end{equation}
where $\Omega=(\alpha_{\Xi},\beta_\Xi,\gamma_\Xi,
\alpha_{\bar\Xi},\beta_{\bar\Xi},\gamma_{\bar\Xi}, \alpha_{\Lambda},
\alpha_{\bar\Lambda})$ and
$\boldsymbol{\xi}=(\theta_\Lambda,\varphi_\Lambda,\theta_{\bar\Lambda},\varphi_{\bar\Lambda},\theta_p,\varphi_p,\theta_{\bar
  p},\varphi_{\bar p})$. With the information provided above the explicit
form of the matrices $C_{\mu\bar\nu}$ (Eqn.~\eqref{eqn:c1212}) and
$a_{\mu\nu}$ (Eqn.~\eqref{eqn:matrixa}) it is straightforward to
calculate the joint angular distribution using Eqn.~\eqref{eq:XiXi1},
but the expressions are lengthy.
We find that from $4^4=256$ possible terms, 100 terms are non-zero.
The relevant $Y_{\mu\bar\nu}$ elements (compare non-zero terms
in Eqn.~\eqref{eqn:c1212} ) are: $Y_{00}$, $Y_{02}$, $Y_{11}$, $Y_{13}$, $Y_{20}$, $Y_{22}$, $Y_{31}$, $Y_{33}$. For example:
\begin{equation*}
Y_{00}=({\alpha_\Xi} {\alpha_\Lambda} \cos {\theta_\Lambda}+1) ({\alpha_{\bar\Xi}} {\alpha_{\bar\Lambda}} \cos{\theta_{\bar\Lambda}}+1)  .
\end{equation*}



\section{Data Sample}

\subsection{BESIII Detector}
The BESIII detector~\cite{Ablikim:2009aa} records symmetric $e^+e^-$ collisions
provided by the BEPCII storage ring~\cite{Yu:IPAC2016-TUYA01}, which operates with a peak luminosity of $1\times10^{33}$~cm$^{-2}$s$^{-1}$
in the center-of-mass energy range from 2.0 to  4.95~GeV.
BESIII has collected large data samples in this energy region~\cite{Ablikim:2019hff}. The cylindrical core of the BESIII detector covers 93\% of the full solid angle and consists of a helium-based
 multilayer drift chamber~(MDC), a plastic scintillator time-of-flight
system~(TOF), and a CsI(Tl) electromagnetic calorimeter~(EMC),
which are all enclosed in a superconducting solenoidal magnet
providing a 1.0~T  (0.9~T in
2012) magnetic field. The solenoid is supported by an
octagonal flux-return yoke with resistive plate counter muon
identification modules interleaved with steel.
%The acceptance of charged particles and photons is 93\% over $4\pi$ solid angle.
The charged-particle momentum resolution at $1~{\rm GeV}/c$ is
$0.5\%$, and the
${\rm d}E/{\rm d}x$
resolution is $6\%$ for electrons
from Bhabha scattering. The EMC measures photon energies with a
resolution of $2.5\%$ ($5\%$) at $1$~GeV in the barrel (end cap)
region. The time resolution in the TOF barrel region is 68~ps, while
that in the end cap region is 110~ps.  The end cap TOF
system was upgraded in 2015 using multigap resistive plate chamber
technology, providing a time resolution of
60~ps~\cite{etof}.


\subsection{Data sample}
From 2009 to 2019, over $10^{10}$ $J/\psi$ events were collected with BESIII detector. 
In 10 years, the data sets were taken at four separate time regions, donated 2009, 2012,
2018 and 2019.
The number of $J/\psi$ events is determined by using inclusive decay of the $J/\psi$.
Table~\ref{tab:jpsievent} shows the number of $J/\psi$ events in each data sets. Due to the variation of 
detector status and reconstruction efficiency, we will perform the analysis for each data 
sets separately.

\begin{table}[hbtp]
	\centering
	\normalsize
	\caption[]{{The number of events for $J/\psi$ data sets. }}
	\label{tab:jpsievent}
\begin{tabular}{lr}
\firsthline
Data sets    & Number of events \\
\hline
	2009 & $(224.0\pm1.3)\times10^6$ \\
	2012 & $(1088.5\pm4.4)\times10^6$ \\
	2018 & \multirow{2}{*}{$(8774.0\pm39.4)\times10^6$} \\
	2019 &  \\
	\hline
	Total & $(1088.5\pm4.4)\times10^6$ \\
\lasthline
\end{tabular}
\end{table}


\subsection{Monte Carlo Simulation}
\label{sec:mcsimulation}
\subsubsection{Inclusive Monte Carlo}

Simulated data samples produced with a {\sc
geant4}-based~\cite{geant4} Monte Carlo (MC) package, which
includes the geometric description of the BESIII detector and the
detector response, are used to determine detection efficiencies
and to estimate backgrounds. The simulation models the beam
energy spread and initial state radiation (ISR) in the $e^+e^-$
annihilations with the generator {\sc
kkmc}~\cite{ref:kkmc}. 

\begin{itemize}
\item {  the $J/\psi$ data set \\
The inclusive MC sample includes both the production of the $J/\psi$
resonance and the continuum processes incorporated in {\sc
		kkmc}~\cite{ref:kkmc}.}
\end{itemize}
All particle decays are modelled with {\sc
evtgen}~\cite{ref:evtgen} using branching fractions 
either taken from the
Particle Data Group~\cite{pdg}, when available,
or otherwise estimated with {\sc lundcharm}~\cite{ref:lundcharm}.
%{\it [ORIGINAL:
%The known decay modes are modelled with {\sc
%evtgen}~\cite{ref:evtgen} using branching fractions taken from the
%Particle Data Group~\cite{pdg}, and the remaining unknown charmonium decays
%are modelled with {\sc lundcharm}~\cite{ref:lundcharm}.] }
Final state radiation~(FSR)
from charged final state particles is incorporated using the {\sc
photos} package~\cite{photos}.

\subsubsection{Signal Monte Carlo}
The following Monte Carlo samples are alse been generated by ourselves.

Phase space (PHSP MC) for two decay channel were generated for calculating 
the normalization in the maximum log likelihood method.

Signal MC samples simulated using the parameters estimated from data (mDIY MC)
were generated as a control sample searching for inconsistencies between
data and MC and used for input/output check and selection criteria optimization.
These values  are within the fit uncertainties of the experimentally obtained 
values and CP-conservation is assumed. The true distributions of the momentum of 
final states and 16 production moments are ploted in Fig.~\ref{fig:tru16momxixipm},~\ref{fig:tru16momxixipp},~\ref{fig:trufmomxixipm},~\ref{fig:trufmomxixipp}.
(For the four sets of MC simulation, the distributions are very similar. We only use 2009 MC sets as 
an example to show the distributions. The distributions of others can be found in Appendix.~\ref{app:trueinfo}.)

\begin{figure*}[hp]
  \centering
  \mbox
  {
  \begin{overpic}[width=0.8\textwidth]{Figure/xixipmmom2009.eps}
  \end{overpic}
  }
	\caption{Using 2009 mDIY MC as example to show the true distributions of the 16 moments of neutron channel for $J/\psi\to \Xi^- \bar{\Xi}^+$.} 
 \label{fig:tru16momxixipm}
\end{figure*}



\begin{figure*}[hp]
  \centering
  \mbox
  {
  \begin{overpic}[width=0.8\textwidth]{Figure/xixippmom2009.eps}
  \end{overpic}
  }
 \caption{Using 2009 mDIY MC as example to show the true distributions of the 16 moments of anti-neutron channel for $J/\psi\to \Xi^- \bar{\Xi}^+$.}
 \label{fig:tru16momxixipp}
\end{figure*}


\begin{figure*}[hp]
  \centering
  \mbox
  {
  \begin{overpic}[width=0.8\textwidth]{Figure/xixipmfmom2009.eps}
  \end{overpic}
  }
 \caption{Using 2009 mDIY MC as example to show the true momentum of final states and $\Lambda$ resonances in neutron channel for $J/\psi\to \Xi^- \bar{\Xi}^+$ The order of final states in legend is 
	$\Lambda$, $\bar{\Lambda}$, $n$, $\bar{p}$, $\pi^+_{\Xi}$, $\pi^+_{\Lambda}$, $\pi^-_{\Xi}$, and $\pi^0$.}
 \label{fig:trufmomxixipm}
\end{figure*}

\begin{figure*}[hp]
  \centering
  \mbox
  {
  \begin{overpic}[width=0.8\textwidth]{Figure/xixippfmom2009.eps}
  \end{overpic}
  }
 \caption{Using 2009 mDIY MC as example to show the true momentum of final states and $\Lambda$ resonances in anti-neutron channel for $J/\psi\to \Xi^- \bar{\Xi}^+$ The order of final states in legend is 
	$\Lambda$, $\bar{\Lambda}$, $p$, $\bar{n}$, $\pi^-_{\Xi}$, $\pi^-_{\Lambda}$, $\pi^+_{\Xi}$, and $\pi^0$.}
 \label{fig:trufmomxixipp}
\end{figure*}




The number of events of PHSP MC and mDIY MC samples are decided according to
the number of $J/\psi$ events and the branching fraction 
$\mathcal{B}(J/\psi \to \Xi^-\bar{\Xi}^+) = (9.7\pm0.8)\times10^{-4}$, 
$\mathcal{B}(\Xi^- \to \Lambda \pi^-) = (99.887\pm0.035)\%$,
$\mathcal{B}(\Lambda \to p \pi^-) = (63.9\pm0.5)\%$, and
$\mathcal{B}(\Lambda \to n \pi^0) = (35.8\pm0.5)\%$.
There is no doubt that the more statistic the MC has, the better for analysis.
Taking CPU time into account, we decide to generate a PHSP MC sample and a mDIY 
MC sample with 30 times the corresponding experimental data statistic. The number of 
event of MC samples for differents years are listed in Table.~\ref{tab:mcevent}.
\begin{table}[hbtp]
	\centering
	\normalsize
	\caption[]{{The number of events for PHSP MC and mDIY MC samples. }}
	\label{tab:mcevent}
\begin{tabular}{lrr}
\firsthline
	Data sets    & PHSP MC (million) & mDIY MC (million)  \\
\hline
	2009 & 1.8 & 1.8 \\
	2012 & 9 & 9 \\
	2018 & 37.8 & 37.8 \\
	2019 & 37.8 & 37.8 \\
	\hline
	Total & 86.4 & 86.4 \\
\lasthline
\end{tabular}
\end{table}




\section{Event Selection}
\label{sec:eventselection}
\subsection{Quick review}

\subsection{Track level selection}
\begin{itemize}
\item {\bf Good charged track}
\begin{itemize}
\item Charged tracks detected in the MDC are required to be within a polar angle ($\theta$) range of $|\rm{cos\theta}|<0.93$, where $\theta$ is defined with respect to the $z$-axis,
which is the symmetry axis of the MDC.

\item Due to the long life time of $\Xi$ and $\Lambda$, for charged tracks in final states,
	the distance of closest approach to the interaction point (IP)
must be less than 30\,cm
along the $z$-axis, $|V_{z}|$,
and less than 10\,cm
in the transverse plane, $|V_{xy}|$.
\item The total number of charged tracks should be equal to 4. This requirement can help us 
	reducing the background caused by charged channel
		$J/\psi\to \Xi^- \bar{\Xi}^+ \to \Lambda(\to p\pi^-)\pi^- \bar{\Lambda}(\to \bar{p} \pi^+) \pi^+$.
\end{itemize}
\item {\bf Good photon selection}
\begin{itemize}
  \item Photon candidates are identified using showers in the EMC.  The deposited energy of each shower must be more than 25~MeV in the barrel region ($|\cos \theta|< 0.80$) and more than 50~MeV in the end cap region ($0.86 <|\cos \theta|< 0.92$).  

\item  
To exclude showers that originate from
charged tracks,
the angle subtended by the EMC shower and the position of the closest charged track at the EMC
must be greater than 20 degrees as measured from the IP. 
%[ORIGINAL:
%To exclude showers that originate from
%charged tracks, the angle between the position of each shower in the EMC and the closest extrapolated charged track must be greater than 10 degrees.]

\item  To suppress electronic noise and showers unrelated to the event, the difference between the EMC time and the event start time is required to be within 
[0, 700]\,ns.

\item 
	There is a additional requirement for neutron channel. To veto the shower deposited by 
	neutrons in the EMC, the openning angle between the direction of $\Lambda$ and the 
		direction of a photon shower must be greater than 15 degrees. However, 
		anti-neutron channel doesn't need such requirement. It will be discussed in the following 
		sub-section.
\end{itemize}

\item {\bf Particle identification}
	\begin{itemize}
		\item  One proton/anti-proton, two $\pi^-$/$\pi^+$, and one $\pi^+$/$\pi^-$ must be identified 
			for neutron/anti-neutron channel. According to the true information as shown in Fig.~\ref{fig:trufmomxixipm},~\ref{fig:trufmomxixipp},
			the proton and pion candidates must have momenta
			$p_{pr} > 0.32 \GeV/c$ and $p_{\pi} < 0.30 \GeV/c$, respectively.
			There is no overlap between proton and pion momenta. The comparison
			of reconstruction momenta between data and MC are shown in
			Fig.~\ref{}~\ref{}.
			The alternative would be to use particle identification methods. 
			As discussed in~\cite{}, this method is not considered viable.
	\end{itemize}

\end{itemize}
\subsection{Event level selection}
For both neutron channel and anti-neutron channel, there is two legs, a charged leg 
$\Xi^- \to \Lambda(\to p\pi^-) \pi^-$ or $c.c.$ and a neutral leg
$\bar{\Xi}^+ \to \bar{\Lambda} (\to \bar{n} \pi^0)\pi^+$ or $c.c.$.
A so called single tag double tag method are used to reconstruct the decay process 
from the available pool of charged tracks and photon showers. Single tag is used to 
reconstruct the charged leg and double tag is used to reconstruct the neutral lag.
\begin{itemize}
	\item {\bf Single tag}
		\begin{itemize}
			\item The $\Lambda$ candidate is reconstructed from proton 
				and charged pion and required to pass a primary 
				vertex fit. 
			\item The $\Xi$ candidate reconstructed from the $\Lambda$
				and the remaining pion is required to pass a primary
				and a secondary vertex fit. The secondary vertex fit
				for the $\Lambda$ is set at the decay point of $\Xi$, 
				for the formed $\Xi$ is set at the  interaction point.
			\item The charged combination is selected to minimize 
				$\Delta m_{\Xi \Lambda} = ((m_{p\pi\pi} - m_{\Xi})^2
				+ (m_{p\pi} - m_{\Lambda})^2)^{1/2}$, where $m_{p\pi\pi}$
				and $m_{p\pi}$ denote the reconstructed invariant masses
				of the proton-pion-pion and proton-pion systems, respectively
				and $m_{\Xi}$, $m_{\Lambda}$ are the PDG tabulated masses of 
				$\Xi$ and $\Lambda$, respectively.
		\end{itemize}
	\item {\bf Double tag}
		\begin{itemize}
			\item The $\pi^0$ candidate is reconstructed from a pair of photons
				which survive the good photon selections. An unconstrained 
				mass $M(\gamma\gamma)$ is calculated from energies and momenta
				of two photon pairs and it must be within $M(\pi^0) - 0.06 < 
				M(\gamma\gamma) < M(\pi^0) + 0.04 $
				. A kinematic fit of the two photons
				is also performed with a constraint $M(\gamma\gamma) = M(\pi^0)$.
				The $chi^2$ from kinematic fit must be less than 25.
				And the resulting energies and momenta of $\pi^0$ is saved for 
				further analyses.
			\item A kinematic fit with the following constraints is performed to suppress
				background and improve the resolution especially for final states in
				neutral leg,

				\begin{subequations}
					\begin{align}
						P_{J/\psi} = P_{\Xi^{\pm}} + P_{\pi^{\mp} + P_{\gamma 1}}
						+ P_{\gamma 2} + P_{n/\bar{n}}, \\
						M(\pi^0) = M(\gamma_1 \gamma_2),\\ 
						M(\Lambda/\bar{\Lambda}) = M(\gamma_1 \gamma_2 n/\bar{n}),
					\end{align}
				\end{subequations}
				where $P$ stands for the four-momenta, $M$ stands for the invariant mass,
				$\gamma_1$ and $\gamma_2$ is the photon from $\pi^0$ in energy descending order.
				The $\chi^2$ from the kinematic fit is required to be less than 200.
		\end{itemize}
	\item {\bf An additional requirement for neutron channel}
		\begin{itemize}
			\item As mentioned before, the shape of neutron shower is very similar with 
				one of a photon shower. The neutron shower has a possibility to pass 
				the good photon selections and form a $\pi^0$ candidate with a small
				energy photon. As shown in Fig.~\ref{fig:angleLNT}, the true opening 
				angle between $\Lambda$ and $n$ (or, $\bar{\Lambda}$ and $\bar{n}$)
				is less then 0.25~rad. Figure~\ref{fig:angleLN} show the open angle
				between $\Lambda$ and photons after reconstruction. For the neutron 
				channel, the peaks at 0.15~rad is caused by neutron shower being mistaken
				for photon shower. There is no such problem for anti-neutron 
				channel. In order to void the mistaken of neutron showers, an angle 
				cut for $\Lambda$ and the EMC showers is required. 
				When we select photons from the EMC shower queues,
				with four momenta of $\Lambda$ which is obtained by the energy-momentum 
				conservation in the recoiling system of $\bar{\Xi}$ and $\pi^-$, the 
				open angle between $\Lambda$ and the EMC showers can be calculated and
				will be required to be less than $15^\circ$.
		\end{itemize}

	\begin{figure*}[hp]
		\centering
		\mbox
		{
			\begin{overpic}[width=0.8\textwidth]{Figure/angleLNT.eps}
			\end{overpic}
		}
		\caption{}
		\label{fig:angleLNT}
	\end{figure*}

	\begin{figure*}[hp]
		\centering
		\mbox
		{
			\begin{overpic}[width=0.8\textwidth]{Figure/angleLN.eps}
			\end{overpic}
		}
		\caption{}
		\label{fig:angleLN}
	\end{figure*}





\end{itemize}

\subsection{Further selection}
After the initial selection criteria one is left with a sample that needs to 
be polished further. In order to reduce background contributions and reduce
data-Monte Carlo discrepancies, the following selection criteria have been 
applied. 
\begin{itemize}
	\item Further selection criteria
		\begin{itemize}
			\item requiring that the reconstructed decay lengths
				of all final state hyperons are greater than
				0;
			\item requiring that the cosine of the angle of the 
				reconstructed $\Xi^-$ in the center-of-mass
				frame, $\cos\theta_{\Xi^-, {\rm CM}}$, 
				fulfills the requirement $|\cos\theta_{\Xi^-, {\rm CM}}| < 0.84$.
			\item setting a mass window selection criteria for the
				$\Lambda$ candidates. For the neutron channel,
				we require that $|M(\bar{p}\pi^+)| < 0.0115~\GeV/c^2$, 
				for the anti-neutron channel,
				$|M(p\pi^-)| < 0.0115~\GeV/c^2$.
			\item setting a mass window selection criteria for the 
				$\Xi^-$ and $\bar{\Xi}^+$ candidates. For the 
				neutron channel, we require that
				$|M(n\gamma\gamma\pi^-)| < 0.011~\GeV/c^2$,
				$|M(\bar{p}\pi^+\pi^+)| < 0.011~\GeV/c^2$;
				for the anti-neutron channel,
				$|M(\bar{n}\gamma\gamma\pi^+)| < 0.011~\GeV/c^2$,
				$|M(p\pi^-\pi^-)| < 0.011~\GeV/c^2$.
		\end{itemize}
\end{itemize}







\section{Background estimation}
\label{sec:background}
\subsection{Inclusive MC}
The potential backgrounds from other decay processes that might be present
in the data are studied by analyzing the official inclusive MC sample.
After final events selection, a topology method is used to classify the 
survived events. Table.~\ref{tab:topopm2018}~\ref{tab:topopp2018} list the 
dominantly contributing processes
for two decay channels and four data sets. Three variables $M(\Xi^-)$,
$M(\bar{\Xi}^+)$ and $M(n/\bar{n})$ are used to separate the signal 
and background. If a background channel has very similar shapes in these 
three mass spectra with the signal shapes, it is called a peaking background.
They can be categorized into three groups: charged decay channel, 
peaking background, and non-peaking background.



\begin{table}[hbtp]
	\centering
	\normalsize
	\caption[]{}
	\label{tab:topopm2018}
	\begin{tabular}{clcccc}
		\hline \hline
		No. & Decay tree & 2009 & 2012 & 2018 & 2019 \\
		\hline
		1 &  $
		J/\psi \rightarrow \eta_{c} \gamma ,
		\eta_{c} \rightarrow \bar{\Xi}^{+} \Xi^{-} ,
		\bar{\Xi}^{+} \rightarrow \pi^{+} \bar{\Lambda} ,
		\Xi^{-} \rightarrow \pi^{-} \Lambda ,
		\bar{\Lambda} \rightarrow \pi^{+} \bar{p} ,
		$ & 28 & 178 & 611 &535  \\
		2 & $
		J/\psi \rightarrow \pi^{+} \pi^{-} \Lambda \bar{\Lambda} ,
		\Lambda \rightarrow \pi^{0} n ,
		\bar{\Lambda} \rightarrow \pi^{+} \bar{p}
		$ & 23 & 108 & 429 & 348 \\
		3 & $
		J/\psi \rightarrow \bar{\Xi}^{+} \Xi^{-} ,
		\bar{\Xi}^{+} \rightarrow \pi^{+} \bar{\Lambda} ,
		\Xi^{-} \rightarrow \pi^{-} \Lambda ,
		\bar{\Lambda} \rightarrow \pi^{+} \bar{p} ,
		\Lambda \rightarrow \pi^{-} p
		$ & 4 & 21 & 86 & 113 \\
		4 & Others & 20 & 197 & 434 & 340 \\
		\hline \hline
	\end{tabular}
\end{table}

\begin{table}[hbtp]
	\centering
	\normalsize
	\caption[]{}
	\label{tab:topopp2018}
	\begin{tabular}{clcccc}
		\hline \hline
		No. & Decay tree & 2009 & 2012 & 2018 & 2019 \\
		\hline
		1 & $
		J/\psi \rightarrow \eta_{c} \gamma ,
		\eta_{c} \rightarrow \bar{\Xi}^{+} \Xi^{-} ,
		\bar{\Xi}^{+} \rightarrow \pi^{+} \bar{\Lambda} ,
		\Xi^{-} \rightarrow \pi^{-} \Lambda ,
		\bar{\Lambda} \rightarrow \pi^{0} \bar{n} ,
		$  & 30 & 175 & 612 &345  \\
		2 & $
		J/\psi \rightarrow \pi^{+} \pi^{-} \Lambda \bar{\Lambda} ,
		\Lambda \rightarrow \pi^{-} p ,
		\bar{\Lambda} \rightarrow \pi^{0} \bar{n}
		$ & 17 & 114 & 373 & 253 \\
		3 & $
		J/\psi \rightarrow \bar{\Xi}^{+} \Xi^{-} ,
		\bar{\Xi}^{+} \rightarrow \pi^{+} \bar{\Lambda} ,
		\Xi^{-} \rightarrow \pi^{-} \Lambda ,
		\bar{\Lambda} \rightarrow \pi^{+} \bar{p} ,
		\Lambda \rightarrow \pi^{-} p
		$ & 6 & 20 & 200 & 149 \\
		4 & Others & 18 & 123 & 431 & 235 \\
		\hline \hline
	\end{tabular}
\end{table}



The charged decay channel has been well studied. The mDIY MC samples for 
charged decay channel are generated according to \cite{}. The mass spectra
of $M(\Xi^-)$, $M(\bar{\Xi}^+)$ and $M(n/\bar{n})$ for this channel are 
ploted in Fig.~\ref{fig:bg1mass}. Figure~\ref{fig:bg1MvM} shows the 2D 
mass distribution $M(\Xi^-)$ v.s. $M(\bar{\Xi}^+)$.
This background will be subtracted in the maximum 
log likelihood fit. 

\begin{figure*}[hp]
	\centering
	\mbox
	{
		\begin{overpic}[width=1.0\textwidth]{Figure/bg1mass.eps}
		\end{overpic}
	}
	\caption{}
	\label{fig:bg1mass}
\end{figure*}

\begin{figure*}[hp]
	\centering
	\mbox
	{
		\begin{overpic}[width=0.8\textwidth]{Figure/bg1MvM.eps}
		\end{overpic}
	}
	\caption{}
	\label{fig:bg1MvM}
\end{figure*}


The exclusive decay $J/\psi \to \eta_c\gamma \to\Xi^- \bar{\Xi}^+\gamma$
is a peaking background. According to PDG, the related  branch fractions 
are $\mathcal{B}(J/\psi \to \eta_c\gamma) = (17\pm0.4)\%$ and
$\mathcal{B}(\eta_c \to \Xi^- \bar{\Xi}^+) = (9.0\pm2.6)\times10^4$.
The process $e^-e^+ \to J/\psi \to \eta_c\gamma$ has a angular distribution
$1+\cos^2\theta$. The spin density matrix for $\eta_c \to \Xi^- \bar{\Xi}^+$
is diag(1, -1, 1, 1). The decay matrix for the hyperons have been discussed 
in Sec.~\ref{}. A mDIY MC sample with 30 number of events are 
generated to estimate the background in the data.

For the phase space decay $J/\psi \to \pi^+\pi^- \Lambda \bar{\Lambda}$
and the rest decay channels, figure~\ref{fig:bg3mass}~\ref{fig:bg3MvM} 
show the distributions
of variables $M(\Xi^-)$, $M(\bar{\Xi}^+)$ and $M(n/\bar{n})$. Since the 
shapes for $M(\Xi^-)$ and $M(\bar{\Xi}^+)$ are continuous and flat, a 
sideband method will be used to estimate this background.

\begin{figure*}[hp]
	\centering
	\mbox
	{
		\begin{overpic}[width=1.0\textwidth]{Figure/bg3mass.eps}
		\end{overpic}
	}
	\mbox
	{
		\begin{overpic}[width=1.0\textwidth]{Figure/bg4mass.eps}
		\end{overpic}
	}
	\caption{}
	\label{fig:bg3mass}
\end{figure*}

\begin{figure*}[hp]
	\centering
	\mbox
	{
		\begin{overpic}[width=0.45\textwidth]{Figure/bg3MvM.eps}
		\end{overpic}
	}
	\mbox
	{
		\begin{overpic}[width=0.45\textwidth]{Figure/bg4MvM.eps}
		\end{overpic}
	}
	\caption{}
	\label{fig:bg3MvM}
\end{figure*}

\subsection{Miscombination background}

When the $\pi^0$ are reconstructed from the EMC photon shower queues, there 
is a probability that one or both of the photon showers are noise or fake
photons. That event might survive the final event selection. Althought it is a 
real signal event, the miscombination of photon will lead to a false four
momenta of $\pi^0$ and a wrong angular reconstruction of neutron or anti-neutron.
A photon matched angle is used to judge if the reconstructed photon is
correct or not, which is defined as following: there are two photons in the MC 
truth, $\gamma^1_{\rm true}$ and $\gamma^2_{\rm true}$, and two reconstructed
photons $\gamma^1_{\rm rec}$ and $\gamma^2_{\rm rec}$. If $\theta(\gamma^1_{\rm true},
\gamma^1_{\rm rec}) + \theta( \gamma^2_{\rm true} , \gamma^2_{\rm rec}) < 
\theta(\gamma^1_{\rm true}, \gamma^2_{\rm rec}) + \theta( \gamma^2_{\rm true} , 
\gamma^1_{\rm rec})$, then, the matched angle for photon 1 and photon 2 is 
$\theta^1_{\rm match} = \theta(\gamma^1_{\rm true}, \gamma^1_{\rm rec})$ and
$\theta^2_{\rm match} = \theta(\gamma^2_{\rm true}, \gamma^2_{\rm rec})$, 
respectively; else 
$\theta^1_{\rm match} = \theta(\gamma^1_{\rm true}, \gamma^2_{\rm rec})$,
$\theta^2_{\rm match} = \theta(\gamma^2_{\rm true}, \gamma^1_{\rm rec})$.
Figure~\ref{fig:gamangle} show the distributions of the photon matched angle.
It's obvious that the matched angle for some photons are too large to be 
correct. At the stage of MC study, requirements on photon matched angle
$\theta^1_{\rm match} < 0.3$~rad and $\theta^2_{\rm match} < 0.3$~rad can
be applied to separate the signal and mis-combination background. As shown
in Fig.~\ref{fig:recmass}, the shape of mis-combination background are relatively 
flat compared to one of the signal. It can be used as a variable to distinguish
between signal and background. It is also expectable that the shapes of 
signal and background are very similar for $M(\Xi^-)$, $M(\bar{\Xi}^+)$
and $M(\Lambda)$.
\begin{figure*}[hp]
	\centering
	\mbox
	{
		\begin{overpic}[width=0.9\textwidth]{Figure/gamangle.eps}
		\end{overpic}
	}
	\caption{}
	\label{fig:gamangle}
\end{figure*}
\begin{figure*}[hp]
	\centering
	\mbox
	{
		\begin{overpic}[width=0.9\textwidth]{Figure/recmass.eps}
		\end{overpic}
	}
	\caption{}
	\label{fig:recmass}
\end{figure*}
\begin{figure*}[hp]
	\centering
	\mbox
	{
		\begin{overpic}[width=0.9\textwidth]{Figure/baryonangle.eps}
		\end{overpic}
	}
	\caption{}
	\label{fig:baryonangle}
\end{figure*}

Figure~\ref{fig:baryonangle} show that the mis-combination background will 
lead to a worse resolution of the reconstruction of neutron/anti-neutron 
position. A large mDIY MC sample will be used to estimate the mis-combination
background.

\section{Parameters estimation}
\subsection{Maximum likelihood fit}

A maximum likelihood fit is used to extract the decay parameters, which are only 
related to the helicity angles. For each channel, the probability distribution 
function (PDF) with eight unknow parameters $\Omega$ can be defined in nine dimensions
helicity angles spanned by $\xi$:
\begin{equation}
	\mathcal{P}(\xi;\Omega) = \mathcal{W}(\xi;\Omega)\varepsilon(\xi)/\mathcal{N}(\Omega)
\end{equation}
The full likelihood function can be written as 
\begin{equation}
	\mathcal{S} = -\sum_{i = 1}^N\ln \mathcal{W}(\xi_i;\Omega) + 
	\sum_j \sum_i^{N_j^{\rm bkg}}\ln \mathcal{W}(\xi_i;\Omega)
	+ N^{\rm signal} \times \ln \mathcal{N}(\Omega) - \sum_i^N \ln \varepsilon(\xi_i)
\end{equation}

\begin{itemize}
	\item Helicity angles and parameters

		For neutron channel, the helicity angles $\xi = (\theta_{\Xi}, \theta_{\Lambda}, \phi_{\Lambda}, 
		\theta_{\bar{\Lambda}}, \phi_{\bar{\Lambda}}, \theta_{n}, \phi_{n}, 
		\theta_{\bar{p}}, \phi_{\bar{p}})$ 
		and the parameters $\Omega = (\alpha_{J/\psi}, \Delta \Phi, \alpha_{\Xi^-},
		\phi_{\Xi^-}, \alpha_{\bar{\Xi}^+}, \phi_{\bar{\Xi}^+}, \alpha_{\Lambda}^0,
		\alpha_{\bar{\Lambda}}^+)$;

		For anti-neutron channel, the helicity angles 
		$\xi = (\theta_{\Xi}, \theta_{\Lambda}, \phi_{\Lambda}, 
		\theta_{\bar{\Lambda}}, \phi_{\bar{\Lambda}}, \theta_{p}, \phi_{p}, 
		\theta_{\bar{n}}, \phi_{\bar{n}})$ 
		and the parameters $\Omega = (\alpha_{J/\psi}, \Delta \Phi, \alpha_{\Xi^-},
		\phi_{\Xi^-}, \alpha_{\bar{\Xi}^+}, \phi_{\bar{\Xi}^+}, \alpha_{\Lambda}^-,
		\alpha_{\bar{\Lambda}}^0)$.
	\item $-\sum_{i = 1}^N\ln \mathcal{W}(\xi_i;\Omega)$

		The first item is the sum of all events (total $N$ events) which survive the 
		final selection criteria in experiment data. $\mathcal{W}(\xi_i;\Omega)$ is 
		the amplitude value for $i$-th events. In practice, we will try to find 
		minimum value of a function. So, there is a negative sign in the PDF.
	\item $\sum_j \sum_i^{N_j^{\rm bkg}}\ln \mathcal{W}(\xi_i;\Omega)$

		The second item stands for the background subtracting. There are total
		$j = 4$ types of background: mis-combination background, charged decay
		channel, $J/\psi \to \gamma \eta_c$ and non-peaking background. $N_j^{\rm bkg}$
		is the estimated number of events for $j$-th type of background.

	\item $N^{\rm signal} \times \ln \mathcal{N}(\Omega)$

		$\mathcal{N}(\Omega)$ is the normalization factor given by:
		\begin{equation}
			\mathcal{N}(\Omega) = \int \mathcal{W}(\xi;\Omega)\varepsilon(\xi) {\rm d} \xi 
		\end{equation}
		In practice, there are two method to calculate the normalization factor,
		using PHSP MC or mDIY MC, propagated through the detector and reconstructed as
		data. The normalization factor calculated with PHSP MC is approximately given 
		as:
		\begin{equation}
			\mathcal{N}(\Omega) = \frac{1}{M} \sum_{j=1}^M \mathcal{W}(\xi_j;\Omega)	
		\end{equation}
		By using mDIY MC, the normalization factor is given as:
		\begin{equation}
			\mathcal{N}(\Omega) = \frac{1}{M} \sum_{j=1}^M \frac{\mathcal{W}(\xi_j;\Omega)	}{\mathcal{W}(\xi_j;\Omega_0)}
		\end{equation}
		where $M$ is the total number of events of the corresponding MC sample that  
		survived all the event selection criteria as data and with additional photon 
		matched angle requirement..

		{\bf Comments on normalization factor}

	\item $- \sum_i^N \ln \varepsilon(\xi_i)$

	 	The last item is the efficiencies which is not dependent on the parameters
		in $\Omega$ and will only affects the overall log-likelihood normalization.
		Therefore, this item is a constent and can be dropped.


\end{itemize}

\subsection{Input-Output check}

An input-output check is performed to guarantee the correctness of fitting procedure. 
30 sets of mDIY MC is used to do the input-output check and the number of events of
each set is equal to the signal yield from data. The backgrounds are also considered
by including the background events from the inclusive and exclusive MC and then using 
the same method as used in data to subtract the background. Table~\ref{} lists the
input-output value. The distributions of the fitting results and the pull distribution
are shown in Fig.~\ref{}. It is clear to see that the output values are consistent with
the input.


\section{Real data}
\subsection{Blind analysis}
In order to reduce the experimenter's bias, this measurement
adopts a blind analysis technique~\cite{Roodman:2003rw}. As discussed in 
Sec.~\ref{sec:mcsimulation} and Sec.~\ref{sec:background},
the signal MC and background MC are generated according to the
knowledge that we have learned from previous experiments. The events
selections in Sec.~\ref{sec:eventselection} are optimized according 
to the MC simulation without looking at the data. The total measurement
procedure is fixed on a sub-sample of the data, 2009 + 2012 + 1/3 of 2018 
and 2019 data samples.
The systematic impact of the defferences between MC and data are
studied by applying a hidden answer method. Once all the strategies
are sattled, it will be performed to the full data sets to open the
hidden box. The follows are the comparisons of distribution  
between data and MC which show the agreement of them.

\subsection{Signal yield}
According to the discussion in Sec.~\ref{sec:background}, the invariant mass
of neutron/anti-neutron is chosen as a variable to obtain the signal yield,
with an unbinned maximum likelihood fit. The signal shape is modeled as the
likelihood function by a tool {\sc RooKeysPdf} in {\sc RooFit} to describe  
the signal. As shown in Fig.~\ref{fig:bkg2012xixipm}, the 
shape of the miscombination background drops rapidly from 0.96 to 0.98~GeV/$c^2$.
A standard argus function is not enough to describe it. We choose the product
of an argus function and a 3rd order polynomial to model this background shape.

In the fit of the invariant mass of neutron/anti-neutron, the parameters for 
the background function is fixed by fitting the background shape which is 
from mDIY MC. The blue line in Fig.~\ref{fig:bkg2012xixipm} is the 
parametric background shape. The full likelihood function is the sum of 
background function and signal shape convolved with a Gaussian 
function. The results of fit is show in Fig~\ref{fig:bkg2012xixipm}. 
The background have to 
be subtrack from this fit results. The signal yield is calculated as follow 
equation:
\begin{equation}
	N_{\rm sig} = N_{\rm fit} - N_{\rm sideband} - N_{\rm peaking},
\end{equation}

where $N_{\rm sideband}$ is the background events estimated by a sideband method 
which is defined as shown in Fig.~\ref{fig:xi2dmassdata}. $N_{\rm peaking}$
is the number of events of peaking background channel $J/\psi \to \gamma \eta_c$,
which is estimated with a exclusive MC simulation.


\begin{figure*}[hp]
	\centering
		\mbox
		{
			\begin{overpic}[width=0.45\textwidth]{Figure/2012xixipmv5/modelbkg.eps}
			\end{overpic}
		}
		\mbox
		{
			\begin{overpic}[width=0.45\textwidth]{Figure/2012xixipmv5/massfit.eps}
			\end{overpic}
		}
		\caption{2012 neutron channel}
		\label{fig:bkg2012xixipm}
\end{figure*}


\begin{figure*}[hp]
	\centering
		\mbox
		{
			\begin{overpic}[width=0.45\textwidth]{Figure/2012xixipmv5/xi2dmassdata.eps}
			\end{overpic}
		}
		\caption{2012 neutron channel}
		\label{fig:xi2dmassdata}
\end{figure*}



\subsection{Variables comparison}
The agreement between data and MC of the varibles which are used as 
event selection requirements to suppress the background are necessary
to be checked. Figure~\ref{fig:variables2012xixipm} take the neutron 
channel in 2012 as an example to show that distributions in a order
1) $\bar{\Lambda}$ decay length, 2) $\bar{\Xi}^+$ decay length, 
3) $\bar{\Xi}^+$ invariant mass, 4) $\Xi^-$ invariant mass,
5) $\bar{\Lambda}$ invariant mass, 6) $\cos \theta (\bar{\Xi}^+)$,
7) $n$ invariant mass, 8) the angle between $\Lambda$ and $\gamma_1$,
9) the angle between $\Lambda$ and $\gamma_2$, 
10) $\chi^2$ of secondary vertex fit of $\bar{\Xi}^+$,
11) $\chi^2$ of secondary vertex fit of $\bar{\Lambda}$,
12) $\chi^2$ of kinematic fit.
The distributions of other data samples are very similar to 
the example and they all have a good agreement between data 
and corresponding MC simualtion. We will leave the rests in 
Appendix~\ref{app:variablescomparison}.


\begin{figure*}[hp]
	\centering
	\foreach \n in {1, 2, 3, 4, 5, 6, 7, 8, 9, 10, 11, 12}{
		\mbox
		{
			\begin{overpic}[width=0.31\textwidth]{Figure/2012xixipmv5/can\n.eps}
			\end{overpic}
		}
		}
		\caption{2012 neutron channel}
		\label{fig:variables2012xixipm}
\end{figure*}

\subsection{Transverse momentum and polar angle}
The transverse momenta and polar angle of the resonance
$\Xi$ and $\Lambda$ and the final states proton, neutron,
charge and neutral pion are also ploted to show that 
there is no bias. Figure~\ref{fig:pt2012xixipm} and 
\ref{fig:costheta2012xixipm} present the distributions of
transverse momenta and polar angle for the neutron channel in 2012, respectively.
All the particles in decay chain are listed in order 
1) $\bar{\Xi}^+$, 2) $\Xi^+$, 3) $\Lambda$, 4) $\bar{\Lambda}$,
5) proton, 6) neutron,
7) $\pi^+$ from $\bar{\Xi}^+$, 8) $\pi^+$ from $\bar{\Lambda}$,
9) $\pi^-$ from $\Xi^-$, 10) $\pi^0$, 11) $\gamma_1$, 12) $\gamma_2$.
It has to be mentioned that we plot the distributions of the momentum 
of $\pi^0$ and the energy of $\gamma$ considering of the 
independent freedem of detection. The distributions of the rest 
data samples can be found in Appendix~\ref{app:transverse}.


\begin{figure*}[hp]
	\centering
	\foreach \n in {13, 14, 15, 16, 17, 18, 19, 20, 21, 22, 23, 24}{
		\mbox
		{
			\begin{overpic}[width=0.31\textwidth]{Figure/2012xixipmv5/can\n.eps}
			\end{overpic}
		}
		}
		\caption{2012 neutron channel}
		\label{fig:pt2012xixipm}
\end{figure*}




\begin{figure*}[hp]
	\centering
	\foreach \n in {25, 26, 27, 28, 29, 30, 31, 32, 33, 34, 35, 36}{
		\mbox
		{
			\begin{overpic}[width=0.31\textwidth]{Figure/2012xixipmv5/can\n.eps}
			\end{overpic}
		}
		}
		\caption{2012 neutron channel}
		\label{fig:costheta2012xixipm}
\end{figure*}

\subsection{Polarization and entanglement}
\subsection{Data fit results}


\section{Systematic uncertainty}
The sources of systematic uncertainties are listed as follows:

\begin{itemize}
	\item Reconstruction of $\Lambda$: decay length, chi2?
	\item Reconstruction of $\Xi$: decay length, chi2?
	\item Reconstruction of $\pi^0$; chi2? control sample?
	\item Reconstruction of $\pi^\pm$ control sample?
	\item Kinematic Fit :  chi2 , helix correction
	\item Mass cut : barlow test
	\item Cos theta cut : barlow test
	\item Fitting method : 
	\item Mis-combination background : mDIY and sideband
	\item exclusive Background : mDIY and side band
\end{itemize}

\include{10Summary}

%%%%%%%%%%%%%%%%%%%%%%%%%%%%%%%%%%%%%%%%%%%%%%%%%%%%%%%%%%%%%%%%%%%%%%%%%%%%%%
%%=============================================================================
\clearpage

%\bibliographystyle{besnote}
%\bibliography{Lambdac4600}
\include{bibitem}

%%%%%%%%%%%%%%%%%%%%%%%%%%%%%%%%%%%%%%%%%%%%%%%%%%%%%%%%%%%%%%%%%%%%%%%%%%%%%%
\clearpage
\appendix
\part*{Appendices}
\addcontentsline{toc}{part}{Appendices}
\section{True information}
\label{app:trueinfo}

In order to have an overall impression of the distribution of momentums of
the final states and 16 moments, the true information are ploted to 
provide reference for the determination of selection conditions.
We plot the distributions for 2009, 2012, 2018, and 2019 samples separately.
The statistic for each sample is 30 times larger than the corresponding 
data sample.


\begin{figure*}[hp]
  \centering
  \mbox
  {
  \begin{overpic}[width=0.8\textwidth]{Figure/xixipmmom2009.eps}
  \end{overpic}
  }
	\caption{Using 2009 mDIY MC as example to show the true distributions of the 16 moments of neutron channel for $J/\psi\to \Xi^- \bar{\Xi}^+$.} 
 \label{fig:tru16momxixipm}
\end{figure*}



\begin{figure*}[hp]
  \centering
  \mbox
  {
  \begin{overpic}[width=0.8\textwidth]{Figure/xixippmom2009.eps}
  \end{overpic}
  }
 \caption{Using 2009 mDIY MC as example to show the true distributions of the 16 moments of anti-neutron channel for $J/\psi\to \Xi^- \bar{\Xi}^+$.}
 \label{fig:tru16momxixipp}
\end{figure*}




\begin{figure*}[hp]
  \centering
  \mbox
  {
  \begin{overpic}[width=0.8\textwidth]{Figure/xixipmmom2012.eps}
  \end{overpic}
  }
	\caption{Using 2012 mDIY MC as example to show the true distributions of the 16 moments of neutron channel for $J/\psi\to \Xi^- \bar{\Xi}^+$.} 
 \label{fig:tru16momxixipm}
\end{figure*}



\begin{figure*}[hp]
  \centering
  \mbox
  {
  \begin{overpic}[width=0.8\textwidth]{Figure/xixippmom2012.eps}
  \end{overpic}
  }
 \caption{Using 2012 mDIY MC as example to show the true distributions of the 16 moments of anti-neutron channel for $J/\psi\to \Xi^- \bar{\Xi}^+$.}
 \label{fig:tru16momxixipp}
\end{figure*}


\begin{figure*}[hp]
  \centering
  \mbox
  {
  \begin{overpic}[width=0.8\textwidth]{Figure/xixipmmom2018.eps}
  \end{overpic}
  }
	\caption{Using 2018 mDIY MC as example to show the true distributions of the 16 moments of neutron channel for $J/\psi\to \Xi^- \bar{\Xi}^+$.} 
 \label{fig:tru16momxixipm}
\end{figure*}



\begin{figure*}[hp]
  \centering
  \mbox
  {
  \begin{overpic}[width=0.8\textwidth]{Figure/xixippmom2018.eps}
  \end{overpic}
  }
 \caption{Using 2018 mDIY MC as example to show the true distributions of the 16 moments of anti-neutron channel for $J/\psi\to \Xi^- \bar{\Xi}^+$.}
 \label{fig:tru16momxixipp}
\end{figure*}


\begin{figure*}[hp]
  \centering
  \mbox
  {
  \begin{overpic}[width=0.8\textwidth]{Figure/xixipmmom2019.eps}
  \end{overpic}
  }
	\caption{Using 2019 mDIY MC as example to show the true distributions of the 16 moments of neutron channel for $J/\psi\to \Xi^- \bar{\Xi}^+$.} 
 \label{fig:tru16momxixipm}
\end{figure*}



\begin{figure*}[hp]
  \centering
  \mbox
  {
  \begin{overpic}[width=0.8\textwidth]{Figure/xixippmom2019.eps}
  \end{overpic}
  }
 \caption{Using 2019 mDIY MC as example to show the true distributions of the 16 moments of anti-neutron channel for $J/\psi\to \Xi^- \bar{\Xi}^+$.}
 \label{fig:tru16momxixipp}
\end{figure*}



\begin{figure*}[hp]
  \centering
  \mbox
  {
  \begin{overpic}[width=0.8\textwidth]{Figure/xixipmfmom2012.eps}
  \end{overpic}
  }
 \caption{Using 2012 mDIY MC as example to show the true momentum of final states and $\Lambda$ resonances in neutron channel for $J/\psi\to \Xi^- \bar{\Xi}^+$ The order of final states in legend is 
	$\Lambda$, $\bar{\Lambda}$, $n$, $\bar{p}$, $\pi^+_{\Xi}$, $\pi^+_{\Lambda}$, $\pi^-_{\Xi}$, and $\pi^0$.}
 \label{fig:trufmomxixipm}
\end{figure*}

\begin{figure*}[hp]
  \centering
  \mbox
  {
  \begin{overpic}[width=0.8\textwidth]{Figure/xixipmfmom2012.eps}
  \end{overpic}
  }
 \caption{Using 2012 mDIY MC as example to show the true momentum of final states and $\Lambda$ resonances in anti-neutron channel for $J/\psi\to \Xi^- \bar{\Xi}^+$ The order of final states in legend is 
	$\Lambda$, $\bar{\Lambda}$, $p$, $\bar{n}$, $\pi^-_{\Xi}$, $\pi^-_{\Lambda}$, $\pi^+_{\Xi}$, and $\pi^0$.}
 \label{fig:trufmomxixipp}
\end{figure*}



\begin{figure*}[hp]
  \centering
  \mbox
  {
  \begin{overpic}[width=0.8\textwidth]{Figure/xixipmfmom2018.eps}
  \end{overpic}
  }
 \caption{Using 2018 mDIY MC as example to show the true momentum of final states and $\Lambda$ resonances in neutron channel for $J/\psi\to \Xi^- \bar{\Xi}^+$ The order of final states in legend is 
	$\Lambda$, $\bar{\Lambda}$, $n$, $\bar{p}$, $\pi^+_{\Xi}$, $\pi^+_{\Lambda}$, $\pi^-_{\Xi}$, and $\pi^0$.}
 \label{fig:trufmomxixipm}
\end{figure*}

\begin{figure*}[hp]
  \centering
  \mbox
  {
  \begin{overpic}[width=0.8\textwidth]{Figure/xixipmfmom2018.eps}
  \end{overpic}
  }
 \caption{Using 2018 mDIY MC as example to show the true momentum of final states and $\Lambda$ resonances in anti-neutron channel for $J/\psi\to \Xi^- \bar{\Xi}^+$ The order of final states in legend is 
	$\Lambda$, $\bar{\Lambda}$, $p$, $\bar{n}$, $\pi^-_{\Xi}$, $\pi^-_{\Lambda}$, $\pi^+_{\Xi}$, and $\pi^0$.}
 \label{fig:trufmomxixipp}
\end{figure*}


\begin{figure*}[hp]
  \centering
  \mbox
  {
  \begin{overpic}[width=0.8\textwidth]{Figure/xixipmfmom2019.eps}
  \end{overpic}
  }
 \caption{Using 2019 mDIY MC as example to show the true momentum of final states and $\Lambda$ resonances in neutron channel for $J/\psi\to \Xi^- \bar{\Xi}^+$ The order of final states in legend is 
	$\Lambda$, $\bar{\Lambda}$, $n$, $\bar{p}$, $\pi^+_{\Xi}$, $\pi^+_{\Lambda}$, $\pi^-_{\Xi}$, and $\pi^0$.}
 \label{fig:trufmomxixipm}
\end{figure*}

\begin{figure*}[hp]
  \centering
  \mbox
  {
  \begin{overpic}[width=0.8\textwidth]{Figure/xixipmfmom2019.eps}
  \end{overpic}
  }
 \caption{Using 2019 mDIY MC as example to show the true momentum of final states and $\Lambda$ resonances in anti-neutron channel for $J/\psi\to \Xi^- \bar{\Xi}^+$ The order of final states in legend is 
	$\Lambda$, $\bar{\Lambda}$, $p$, $\bar{n}$, $\pi^-_{\Xi}$, $\pi^-_{\Lambda}$, $\pi^+_{\Xi}$, and $\pi^0$.}
 \label{fig:trufmomxixipp}
\end{figure*}









\include{Appendix_BranchFraction}
\section{Comparison between data and MC}


\subsection{Variables comparison}
\label{app:variablescomparison}

\ref{fig:variables2009xixipp},\ref{fig:variables2012xixipm},\ref{fig:variables2012xixipp}.

\begin{figure*}[hp]
	\centering
	\foreach \n in {1, 2, 3, 4, 5, 6, 7, 8, 9, 10, 11, 12}{
		\mbox
		{
			\begin{overpic}[width=0.31\textwidth]{Figure/2009xixipmv5/can\n.eps}
			\end{overpic}
		}
		}
		\caption{2009 neutron channel}
		\label{fig:variables2009xixipm}
\end{figure*}


\begin{figure*}[hp]
	\centering
	\foreach \n in {1, 2, 3, 4, 5, 6, 7, 8, 9, 10, 11, 12}{
		\mbox
		{
			\begin{overpic}[width=0.31\textwidth]{Figure/2009xixippv5/can\n.eps}
			\end{overpic}
		}
		}
		\caption{2009 anti-neutron channel}
		\label{fig:variables2009xixipp}
\end{figure*}


\begin{figure*}[hp]
	\centering
	\foreach \n in {1, 2, 3, 4, 5, 6, 7, 8, 9, 10, 11, 12}{
		\mbox
		{
			\begin{overpic}[width=0.31\textwidth]{Figure/2012xixipmv5/can\n.eps}
			\end{overpic}
		}
		}
		\caption{2012 neutron channel}
		\label{fig:variables2012xixipmapp}
\end{figure*}


\begin{figure*}[hp]
	\centering
	\foreach \n in {1, 2, 3, 4, 5, 6, 7, 8, 9, 10, 11, 12}{
		\mbox
		{
			\begin{overpic}[width=0.31\textwidth]{Figure/2012xixippv5/can\n.eps}
			\end{overpic}
		}
		}
		\caption{2012 anti-neutron channel}
		\label{fig:variables2012xixipp}
\end{figure*}

\subsection{Transverse momentum and polar angle}
\label{app:transverse}


\begin{figure*}[hp]
	\centering
	\foreach \n in {13, 14, 15, 16, 17, 18, 19, 20, 21, 22, 23, 24}{
		\mbox
		{
			\begin{overpic}[width=0.31\textwidth]{Figure/2009xixipmv5/can\n.eps}
			\end{overpic}
		}
		}
		\caption{2009 neutron channel}
		\label{fig:pt2009xixipm}
\end{figure*}


\begin{figure*}[hp]
	\centering
	\foreach \n in {13, 14, 15, 16, 17, 18, 19, 20, 21, 22, 23, 24}{
		\mbox
		{
			\begin{overpic}[width=0.31\textwidth]{Figure/2009xixippv5/can\n.eps}
			\end{overpic}
		}
		}
		\caption{2009 anti-neutron channel}
		\label{fig:pt2009xixipp}
\end{figure*}


\begin{figure*}[hp]
	\centering
	\foreach \n in {13, 14, 15, 16, 17, 18, 19, 20, 21, 22, 23, 24}{
		\mbox
		{
			\begin{overpic}[width=0.31\textwidth]{Figure/2012xixipmv5/can\n.eps}
			\end{overpic}
		}
		}
		\caption{2012 neutron channel}
		\label{fig:pt2012xixipmpp}
\end{figure*}


\begin{figure*}[hp]
	\centering
	\foreach \n in {13, 14, 15, 16, 17, 18, 19, 20, 21, 22, 23, 24}{
		\mbox
		{
			\begin{overpic}[width=0.31\textwidth]{Figure/2012xixippv5/can\n.eps}
			\end{overpic}
		}
		}
		\caption{2012 anti-neutron channel}
		\label{fig:pt2012xixipp}
\end{figure*}


\begin{figure*}[hp]
	\centering
	\foreach \n in {25, 26, 27, 28, 29, 30, 31, 32, 33, 34, 35, 36}{
		\mbox
		{
			\begin{overpic}[width=0.31\textwidth]{Figure/2009xixipmv5/can\n.eps}
			\end{overpic}
		}
		}
		\caption{2009 neutron channel}
		\label{fig:costheta2009xixipm}
\end{figure*}


\begin{figure*}[hp]
	\centering
	\foreach \n in {25, 26, 27, 28, 29, 30, 31, 32, 33, 34, 35, 36}{
		\mbox
		{
			\begin{overpic}[width=0.31\textwidth]{Figure/2009xixippv5/can\n.eps}
			\end{overpic}
		}
		}
		\caption{2009 anti-neutron channel}
		\label{fig:costheta2009xixipp}
\end{figure*}


\begin{figure*}[hp]
	\centering
	\foreach \n in {25, 26, 27, 28, 29, 30, 31, 32, 33, 34, 35, 36}{
		\mbox
		{
			\begin{overpic}[width=0.31\textwidth]{Figure/2012xixipmv5/can\n.eps}
			\end{overpic}
		}
		}
		\caption{2012 neutron channel}
		\label{fig:costheta2012xixipmapp}
\end{figure*}


\begin{figure*}[hp]
	\centering
	\foreach \n in {25, 26, 27, 28, 29, 30, 31, 32, 33, 34, 35, 36}{
		\mbox
		{
			\begin{overpic}[width=0.31\textwidth]{Figure/2012xixippv5/can\n.eps}
			\end{overpic}
		}
		}
		\caption{2012 anti-neutron channel}
		\label{fig:costheta2012xixipp}
\end{figure*}

\begin{figure*}[hp]
	\centering
		\mbox
		{
			\begin{overpic}[width=0.45\textwidth]{Figure/2009xixipmv5/modelbkg.eps}
			\end{overpic}
		}
		\mbox
		{
			\begin{overpic}[width=0.45\textwidth]{Figure/2009xixipmv5/massfit.eps}
			\end{overpic}
		}
		\caption{2009 neutron channel}
		\label{fig:bkg2009xixipm}
\end{figure*}



\begin{figure*}[hp]
	\centering
		\mbox
		{
			\begin{overpic}[width=0.45\textwidth]{Figure/2009xixippv5/modelbkg.eps}
			\end{overpic}
		}
		\mbox
		{
			\begin{overpic}[width=0.45\textwidth]{Figure/2009xixippv5/massfit.eps}
			\end{overpic}
		}
		\caption{2009 neutron channel}
		\label{fig:bkg2009xixipp}
\end{figure*}



\begin{figure*}[hp]
	\centering
		\mbox
		{
			\begin{overpic}[width=0.45\textwidth]{Figure/2012xixipmv5/modelbkg.eps}
			\end{overpic}
		}
		\mbox
		{
			\begin{overpic}[width=0.45\textwidth]{Figure/2012xixipmv5/massfit.eps}
			\end{overpic}
		}
		\caption{2012 neutron channel}
		\label{fig:bkg2012xixipm}
\end{figure*}



\begin{figure*}[hp]
	\centering
		\mbox
		{
			\begin{overpic}[width=0.45\textwidth]{Figure/2012xixippv5/modelbkg.eps}
			\end{overpic}
		}
		\mbox
		{
			\begin{overpic}[width=0.45\textwidth]{Figure/2012xixippv5/massfit.eps}
			\end{overpic}
		}
		\caption{2012 neutron channel}
		\label{fig:bkg2012xixipp}
\end{figure*}





\end{document}
%%%%%%%%%%%%%%%%%%%%%%%%%%%%%%%%%%%%%%%%%%%%%%%%%%%%%%%%%%%%%%%%%%%%%%%%%%%%%%

